Módulo 1: Dialogflow CX (A Lógica Determinística)
Cenário: Agendamento de Consultas Médicas. Onde se aplica: Quando o processo é rígido e não pode haver erro.

Exemplo de Intent (Intenção):

Frases de Treinamento: "Quero marcar um cardiologista", "Preciso de uma consulta para amanhã", "Tem vaga para doutor Paulo?".

O que acontece: O Dialogflow mapeia essas frases para a intenção agendar_consulta.

Exemplo de Entidades (Entities):

O usuário diz: "Quero ir na unidade Centro na sexta-feira".

O Dialogflow extrai:

@unidade: "Centro"

@sys.date: "2023-10-27" (Converte "sexta-feira" para data ISO automaticamente).

Exemplo de State Machine (Pages):

Página "Coleta de Dados": O bot fica em loop perguntando "Qual o horário?" até que a variável $session.params.horario seja preenchida (Slot Filling).

Rota (Route): Se $session.params.status = "CONFIRMADO", transita para a Página "Fim".

Módulo 1.5: Paradigmas (Híbrido)
Cenário: Assistente Bancário. Onde se aplica: Misturar segurança transacional com facilidade de uso.

Uso Determinístico (Risco Alto):

Fluxo: Transferência PIX.

Por que: Você não quer que uma IA Generativa "improvise" ao perguntar o valor da transferência. O fluxo deve ser exato: Chave -> Valor -> Senha -> Comprovante.

Uso Generativo (Baixo Risco):

Fluxo: Explicação de Termos.

Pergunta: "O que é esse CDB que apareceu na minha conta?"

Ação: O modelo (Vertex AI) gera uma explicação didática baseada no contexto financeiro, sem iniciar nenhuma transação.

A "Mágica" Híbrida:

O usuário diz: "Explica o CDB e investe 100 reais nele".

O Bot usa Generativo para explicar o CDB.

O Bot detecta a intenção de investimento e faz o Hand-off para o fluxo Determinístico que pede a senha.

Módulo 2: GCP Cloud Functions (O Backend)
Cenário: Rastreamento de Encomendas (Logística). Onde se aplica: Conectar o bot aos sistemas da empresa (APIs/Bancos de Dados).

O Gatilho: No Dialogflow, o usuário informa o código "BR12345". O Dialogflow dispara o Webhook para sua URL da Cloud Function.

O Código (Conceitual Node.js):

JavaScript

exports.dialogflowWebhook = async (req, res) => {
  // 1. Extrair o parametro que o Dialogflow enviou
  const trackingCode = req.body.sessionInfo.parameters.codigo_rastreio;

  // 2. Lógica de Negócio: Consultar API da Transportadora
  const status = await shippingApi.getStatus(trackingCode); 
  // Retorno da API: { local: "Curitiba", previsao: "2 dias" }

  // 3. Montar a resposta para o Dialogflow
  res.json({
    fulfillmentResponse: {
      messages: [{
        text: { 
          text: [`Sua encomenda está em ${status.local}. Chega em ${status.previsao}.`] 
        }
      }]
    }
  });
};
Módulo 3: WhatsApp & Gupshup (A Interface)
Cenário: E-commerce de Sapatos. Onde se aplica: Melhorar a UX limitada do WhatsApp.

Problema: O usuário quer ver modelos de tênis. Se o bot responder com texto puro ("Temos Nike, Adidas, Puma..."), fica feio e difícil de ler.

Solução (Custom Payload): Você monta um JSON específico que a Gupshup entende para renderizar uma Lista Interativa.

Exemplo de JSON (Enviado pelo Dialogflow -> Gupshup):

JSON

{
  "type": "list",
  "body": "Escolha sua marca preferida:",
  "action": {
    "button": "Ver Marcas",
    "sections": [
      {
        "title": "Esportivos",
        "rows": [
          { "id": "nike_selection", "title": "Nike Air" },
          { "id": "adidas_selection", "title": "Adidas Ultraboost" }
        ]
      }
    ]
  }
}
Resultado: No WhatsApp, o usuário vê um botão "Ver Marcas", que abre um menu nativo elegante.

Módulo 4: Vertex AI & RAG (A Inteligência)
Cenário: RH / Helpdesk Interno. Onde se aplica: Responder perguntas baseadas em documentos longos (PDFs) sem precisar criar centenas de intents manuais.

O Documento: Um PDF de 50 páginas chamado "Normas da Empresa 2024".

A Pergunta do Usuário: "Posso vir de bermuda na sexta-feira?"

Nota: Não existe uma Intent criada para "bermuda". O Dialogflow tradicional falharia.

Ação do Vertex AI (Search & Conversation):

Recebe a pergunta.

Busca no PDF (via vetores) o trecho que fala sobre "Código de Vestimenta".

Encontra o parágrafo: "Às sextas-feiras, o traje casual é permitido, incluindo bermudas jeans..."

Gera a Resposta: "Sim, de acordo com as normas de vestimenta, o uso de bermudas jeans é permitido às sextas-feiras."

Citação: Adiciona um link para a página 12 do PDF.