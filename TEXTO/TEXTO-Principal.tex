%
% Exemplo genérico de uso da classe unipampa.cls
%
% Se você não tem familiaridade com o LaTeX, este arquivo dá algumas orientações.
% Para um aproveitamento melhor, sugere-se usar o livro LaTeX - A Document Preparation System,
% de Leslie Lamport. Na Internet estão disponíveis também alguns milhares de tutorias.
% Na UNIPAMPA, volta e meia tem cursos, fique atento.
%
% O símbolo % é um comentário de linha, então tudo que aparecer depois dele não é considerado
% no texto final. Você pode limpar todos os comentários deste arquivo, depois que colocar os
% dados corretos, sem prejuízo do texto final.
%

\documentclass[tcc,baec]{unipampa}
% Para usar o modelo, deve-se informar o curso e o tipo de documento e o tipo de documento que deve ser produzido.
% Cursos:
%   * código do curso   -- Usar o código do curso, conforme está registrado no SIE 
%                         (baec para Engenharia de Computação, por exemplo) quando
%                         se tratar de curso de graduação; usar a sigla do programa
%                         quando se tratar de pós-graduação stricto sensu (ppgcap
%                         para o Programa de Pós-graduação em Computação Aplicada,
%                         por exemplo; para especializações, definir os campos
%                         apropriadamente com o comando \course{nome-do-curso}
%                         (sem o termo ``Especialização'') e \campus{nome-do-campus}.
%   
% Tipos de Documento:
%   * tcc               -- Trabalhos de Conclusão de Curso
%   * espec             -- Monografias de Especialização
%   * mestrado          -- Dissertações de Mestrado (acadêmico)
%   * mestradoprof      -- Dissertações de Mestrado (profissional)
%   * doutorado         -- Teses de Doutorado
%   * projetotcc 				-- Projeto de TCC
%   * projetoespec   		-- Projeto de Especialização
%   * projetomestrado		-- Projeto de qualificação de Mestrado
%   * projetodoutorado	-- Projeto de qualificação de Doutorado
%   * relatorio         -- Relatório de projeto (precisa ter o curso de origem e não tem muitos detalhes - trabalho em andamento)
% 
% Outras Opções:
%   * english    -- para textos em inglês
%   * openright  -- força início de capítulos em páginas ímpares (padrão da biblioteca)
%   * oneside    -- desliga frente-e-verso
%   * final      -- versão final do texto
% 

% Programas de pós-graduação com mais de uma área de concentração devem declarar explicitamente
% a área de concentração da dissertação ou tese, por meio do comando
%\renewcommand{\areacourse}{Sanidade Animal}

\usepackage[T1]{fontenc}        % pacote para conj. de caracteres correto
\usepackage[utf8]{inputenc}     % pacote para acentuação
\usepackage{graphicx}           % pacote para importar figuras
\usepackage{times}              % pacote para usar fonte Adobe Times
\usepackage{mathptmx}           % pacote usar fonte Adobe Times nas fórmulas
\usepackage{color}

\usepackage[alf,abnt-emphasize=bf]{abntex2cite}    % pacote para usar citações abnt
\usepackage{amsmath}
\usepackage{amsfonts}
\usepackage{amssymb}
\usepackage{longtable}
\usepackage{textcomp}
\usepackage{listings}
\lstset{
  basicstyle=\ttfamily\small,
  keywordstyle=\color{blue}\bfseries,
  stringstyle=\color{DarkGreen},
  commentstyle=\color{gray}\itshape,
  identifierstyle=\color{black},
  numbers=left,
  numberstyle=\tiny\color{gray},
  stepnumber=1,
  breaklines=true,
  frame=single,
  showstringspaces=false,
  tabsize=2,
  captionpos=b
}
\usepackage[svgnames]{xcolor}
\usepackage{csvsimple}
\usepackage{rotating}
\usepackage{multirow}
\usepackage[final]{pdfpages}

%%%%%%%%%%%% Macros bem jeitosas - a macro obso serve para o seu orientador escrever comentários no
%%%%%%%%%%%% texto, que vão aparecer em azul. A macro \obsa serve para você escrever, e os seus comentários
%%%%%%%%%%%% vão aparecer em laranja. Quando você quiser gerar uma versão sem comentários, comente, com um %
%%%%%%%%%%%% a macro que contém texto e descomente a que não tem. Voilá! Todos os comentários vão desaparecer.

\newcommand{\obso}[1]{\textcolor{blue}{#1}}
%\newcommand{\obso}[1]{}
\newcommand{\obsa}[1]{\textcolor{orange}{#1}}
%\newcommand{\obsa}[1]{}

% Redefine fonte command for left-aligned sources with spacing
\makeatletter
\renewcommand{\fonte}[1]{%
  \par\vspace{0.3em}%
  \begin{flushleft}%
    {\small Fonte: #1}%
  \end{flushleft}%
}
\makeatother


      

% Adjust header height to accommodate the logo
\usepackage{geometry}
\geometry{headheight=2.5cm, headsep=0.5cm}

% Redefine the page style to include the logo in the header
\makeatletter
\renewcommand{\ps@iiufrgs}{
        \let\@oddfoot\@empty
        \let\@evenfoot\@empty
        \def\@evenhead{\hbox to \textwidth{\rlap{\small\thepage}\hfil\includegraphics[height=2cm]{../unipampa.png}\hfil}}
        \def\@oddhead{\hbox to \textwidth{\hfil\includegraphics[height=2cm]{../unipampa.png}\hfil\llap{\small\thepage}}}
}
\makeatother

\begin{document}

\includepdf[pages=-]{primeiras-paginas.pdf}


%%%%%%%%%%%%%%%%%%%%%%%%%%%%%%%%%%%%%%%%%%%%%%%%%%%%%%%%%%%%%%%%%%%%%%%%%%%%
% 
% Aqui comeca o texto propriamente dito. O texto pode ser todo escrito neste 
% mesmo arquivo, mas pode-se separar o texto em diversos arquivos, que podem
% ser incluídos com o comando \input{nome-do-arquivo} (inclui o arquivo com
% nome nome-do-arquivo.tex), que deve estar no mesmo diretório do texto
% principal. Se estiver em outro diretório, pode ser incluído também, usando
% .. (para subir na árvore de diretórios) ou / (para descer), como em
% \input{Textos/nome-do-arquivo}. Dessa forma, o arquivo será buscado no
% subdiretório Textos; se quiser usar caminhos na árvore de diretórios, use
% \input{../Textos/nome-do-arquivo}, que procura o arquivo que está no diretório
% Textos, um nível acima na estrutura.
%
%%%%%%%%%%%%%%%%%%%%%%%%%%%%%%%%%%%%%%%%%%%%%%%%%%%%%%%%%%%%%%%%%%%%%%%%%%%%

% E aqui vai a parte principal:


 


\newpage



\chapter{Referencial Teórico}

As bases técnicas que nortearam a execução das atividades foram a adoção de quatro pilares fundamentais: uma arquitetura de microsserviços, a aplicação dos princípios SOLID, a utilização de um Design System e diferentes linguagens e frameworks. Essa combinação permitiu a entrega de produtos e funcionalidades de forma incremental, consistente e evolutiva, conforme preconizado pelo Manifesto Ágil \cite{agile}.

\section{Conceitos e Metodologias de Desenvolvimento}

Para garantir a qualidade e a manutenibilidade do software, foram adotados os seguintes conceitos e metodologias:

\begin{description}
    \item[Engenharia de Software:] É a aplicação de uma abordagem sistemática, disciplinada e quantificável para o desenvolvimento, operação e manutenção de software \cite{pressman}.
    \item[Metodologia Ágil:] É uma abordagem iterativa para o gerenciamento de projetos e desenvolvimento de software que ajuda as equipes a entregar valor aos seus clientes de forma mais rápida e com agilidade de entrega das funcionalidades.
    \item[Scrum:] É um dos frameworks mais populares para a implementação de desenvolvimento ágil. Ele é projetado para que equipes possam entregar valor ao cliente de forma incremental e iterativa \cite{scrum}.
    \item[Código Limpo (Clean Code):] É uma filosofia de desenvolvimento de software popularizada por Robert C. Martin (``Uncle Bob''). A premissa é que o código deve ser escrito de forma simples, direta e legível, como uma prosa bem escrita. Suas características incluem o uso de nomes significativos para variáveis e funções, funções pequenas com responsabilidades únicas e a minimização de complexidade.
    \item[Arquitetura Limpa (Clean Architecture):] Também concebida por Robert C. Martin, é um modelo de design de software que impõe uma rigorosa separação de responsabilidades. A arquitetura é representada por círculos concêntricos, onde a regra principal é a Regra da Dependência: as dependências do código-fonte só podem apontar para dentro. Nada em um círculo interno pode saber qualquer coisa sobre um círculo externo.
    \item[Princípios SOLID:] É um acrônimo para cinco princípios de design de software que visam tornar o código mais compreensível, flexível e manutenível \cite{cleanarch}:
    \begin{itemize}
        \item \textbf{S} - Single Responsibility Principle
        \item \textbf{O} - Open/Closed Principle
        \item \textbf{L} - Liskov Substitution Principle
        \item \textbf{I} - Interface Segregation Principle
        \item \textbf{D} - Dependency Inversion Principle
    \end{itemize}
    \item[Design Patterns (Padrões de Projeto):] São soluções gerais e reutilizáveis para problemas frequentes no desenvolvimento de software. Popularizados pelo ``Gang of Four'' (GoF), são categorizados em:
    \begin{itemize}
        \item \textbf{Criação:} Abstraem o processo de instanciação de objetos (ex: Factory, Singleton).
        \item \textbf{Estruturais:} Combinam classes e objetos para formar estruturas maiores (ex: Adapter, Decorator).
        \item \textbf{Comportamentais:} Tratam da comunicação e responsabilidades entre objetos (ex: Strategy, Observer).
    \end{itemize}
\end{description}

\section{Arquitetura de Software}

A arquitetura do sistema evoluiu de modelos tradicionais para abordagens mais modernas e distribuídas.

\begin{description}
    \item[Arquitetura Monolítica:] Modelo tradicional onde a aplicação é uma única unidade coesa. Todos os componentes do sitema são fortemente acoplados.
    \item[Arquitetura de Microsserviços:] Padrão onde uma aplicação complexa é composta por serviços pequenos e independentes, cada um com uma capacidade de negócio específica, comunicando-se via Interface de Programação de Aplicações (APIs) \cite{microservices}.
    \item[Computação Serverless:] Permite construir e executar aplicações sem gerenciar servidores.
    \begin{itemize}
        \item \textbf{Cloud Functions:} Funções de propósito único, orientadas a eventos, que escalam automaticamente.
    \end{itemize}
    \item[O Padrão API Gateway:] Atua como ponto único de entrada para requisições, encapsulando a estrutura interna de microsserviços. Seus benefícios incluem desacoplamento, orquestração de dados e centralização de funcionalidades transversais (autenticação, logging).
\end{description}

\section{Tecnologias de Desenvolvimento}

\subsection{Desenvolvimento Web}

\begin{description}
    \item[Node.js:] Ambiente de execução JavaScript (JS) assíncrono construído sobre o motor V8. Seu modelo de I/O não bloqueante o torna ideal para microsserviços e APIs de alta performance \cite{nodejs}.
    \item[React:] Biblioteca JS para construção de interfaces de usuário baseada em componentes e declarativa. Utiliza o Virtual do Document Object Model (DOM) para atualizações eficientes \cite{react}. Conceitos chave incluem Componentes, Reatividade, Gerenciamento de Estados e Hooks.
    \item[Next.js:] Framework React para produção que oferece renderização híbrida (Server Side Rendering e Static Site Generation) e sistema de roteamento baseado em arquivos \cite{nextjs}.
\end{description}

\subsection{Desenvolvimento Mobile}

\begin{description}
    \item[Ecossistema Android:] Sistema operacional móvel baseado em Linux. O desenvolvimento moderno utiliza Kotlin (interoperável com Java) e o sistema de build Gradle.
    \item[Ecossistema iOS:] Sistema proprietário da Apple. O desenvolvimento utiliza Swift (sucessor do Objective-C) e gerenciamento de dependências via Swift Package Manager ou CocoaPods.
    \item[React Native:] Framework para desenvolvimento multiplataforma usando JS e React \cite{reactnative}.
    \begin{itemize}
        \item \textbf{Arquitetura Legada (Bridge):} Baseia-se em uma comunicação assíncrona onde mensagens JavaScript Object Notation  (JSON) são serializadas e enviadas através de uma ``ponte'' entre as threads JS e Nativa. Esse processo de serialização/desserialização introduz um gargalo de performance (overhead), especialmente em aplicações que exigem alta frequência de troca de dados, como animações e gestos.

        \item \textbf{Nova Arquitetura} A nova arquitetura JS Interface (JSI)} representa uma evolução fundamental no React Native. Diferente da arquitetura legada, que dependia de serialização assíncrona de mensagens JSON, a JSI permite que o código JS mantenha referências diretas a objetos C++ (Host Objects). Isso possibilita a invocação síncrona de métodos nativos, eliminando o overhead de serialização e permitindo o compartilhamento de memória entre os runtimes. O resultado é uma interoperabilidade mais rápida e eficiente, essencial para bibliotecas de alta performance e animações complexas. A Figura \ref{fig:com_async} ilustra o processo de criação de um componente e as funções que possibilitam a comunicação entre o código React Native e os módulos nativos.
    \end{itemize}


    \begin{figure}[h!]
        \centering
        \caption{Comunicação JSI entre React-native e Nativo.}
        \label{fig:com_async}
        \includegraphics[scale=0.25]
        {IMAGENS/comunicao-async-rn-2.png} % Placeholder para a imagem
        \vspace{0.3em}
        \newline
        {\small Fonte: Do próprio autor.}
    \end{figure}
    
    \end{itemize}
    \item \textbf{React-native-builder-bob} para criação de bibliotecas e \textit{Upgrade Helper} para migração de versões.
\end{description}

\section{Design e Qualidade de Software}

\begin{description}
    \item[Design System:] Fonte única de verdade para design e desenvolvimento. O Design System provê tokens, componentes e documentação para garantir consistência.
    \item[Chrome DevTools:] Ferramentas de desenvolvimento embutidas no Chrome para inspeção de DOM, depuração de JS e análise de rede/performance.
    \item[Lighthouse:] Ferramenta de auditoria automatizada que avalia Performance, Acessibilidade, Boas Práticas e Otimização para Mecanismos de Busca (SEO).
    \item[Web Vitals:] Iniciativa do Google para métricas de qualidade de experiência do usuário.
    \begin{itemize}
        \item \textbf{Core Web Vitals:} Largest Contentful Paint (LCP), Largest Contentful Paint (CLS) e Interaction to Next Paint (INP).
    \end{itemize}
\end{description}

\section{Inteligência Artificial e Agentes}

\begin{description}
    \item[Agentes Conversacionais:] Evolução do Dialogflow CX, integrando fluxos determinísticos (baseados em estados) e generativos.
    \item[RAG (Retrieval-Augmented Generation):] Técnica que otimiza respostas de Large Language Models (LLM) usando bases de conhecimento confiáveis (Data Stores).
    \item[Model Context Protocol (MCP):] Padrão aberto para conectores seguros entre dados e ferramentas de IA.
    \item[Agent Development Kits (ADK):] Frameworks para construção de agentes autônomos.

  \item[Processamento de Linguagem Natural (NLP):] Trata-se do ramo da Inteligência Artificial que permite às máquinas compreender e gerar a linguagem humana. Unindo computação e linguística, essa tecnologia é a base para o funcionamento eficaz de chatbots, tradutores e assistentes de voz.
\end{description}

\chapter{Atividades Desenvolvidas}

\section{Desenvolvimento de um MVP de um Sistema Web (Período: 30/09 a 25/11)}

\subsection{Detalhamento da Execução}
O MVP está sendo construído e tem previsão de acabar ao longo do final das atividades previstas na disciplina.

\subsection{Atuação em Tarefas}
\textbf{Design e Implementação do API Gateway e do Contrato de Comunicação Inter-serviços.}

A simples criação de endpoints individuais em cada microsserviço levaria a um acoplamento forte com o front-end e a uma complexidade de gerenciamento. Aplicando os princípios de Design de Software, foi implementada uma API Gateway para desacoplamento entre regras de negócios da aplicação.

\subsection{Execução Baseada em Engenharia de Software}
\begin{itemize}
    \item \textbf{Análise de Requisitos:} Em colaboração com a equipe de front-end, mapeei os casos de uso das telas. Isso definiu os dados que o Gateway precisaria agregar.
    \item \textbf{Design da Arquitetura:} Desenhamos o Gateway como um serviço Node.js simples, cuja única responsabilidade (Princípio 'S' do SOLID) era receber requisições do cliente, orquestrar chamadas para os microsserviços internos e agregar as respostas. Isso desacoplou os clientes da topologia interna da nossa rede de serviços.
    \item \textbf{Construção e Qualidade:} Definimos um contrato de comunicação claro usando a especificação OpenAPI (Swagger). Isso permitiu que as equipes de front-end e back-end trabalhassem em paralelo, usando o contrato documentado como fonte da verdade. O Gateway também centralizou a lógica de autenticação de rotas e o tratamento de CORS, simplificando os serviços internos. A implementação seguiu padrões de código limpo para garantir a manutenibilidade futura.
\end{itemize}

\section{Implementação de Componente de Vídeo (Período: 06/10 a 25/11)}

\subsection{Detalhamento da Execução}
O componente está em fase final de desenvolvimento, com a lógica central e os testes unitários concluídos. A etapa final, prevista para as próximas Sprints, é a integração do componente nas telas da aplicação.

\subsection{Atuação em Tarefas}
\textbf{Design, Implementação e Otimização de um Player de Vídeo Reutilizável.}

A criação de um player de vídeo envolve a construção de uma Máquina de Estados Finita (Finite State Machine - FSM) para gerenciar o ciclo de vida do player, a otimização de performance para evitar gargalos de renderização e a garantia de robustez através de uma suíte de testes completa.

\subsection{Execução Baseada em Engenharia de Software}

\subsubsection{Design de Componentes e Lógica de Estado (Hook)}
Seguindo o princípio de Separação de Responsabilidades e a filosofia da Arquitetura Limpa, a lógica de negócio foi completamente isolada da camada de apresentação (UI).

\begin{itemize}
    \item \textbf{Custom Hook (useVideoPlayerState):} Criei um custom hook que serve como o "cérebro" do componente. Ele encapsula toda a interação com a API da biblioteca de vídeo (ex: React Player) e gerencia a FSM interna do player.
    \item \textbf{Estados da FSM:} O hook gerencia os seguintes estados: IDLE, LOADING, PLAYING, PAUSED, SEEKING, ENDED, ERROR.
    \item \textbf{API Exposta pelo Hook:} O hook retorna um objeto bem definido para o componente de UI, contendo:
    \begin{itemize}
        \item \textbf{Estado (state):} Um objeto com informações reativas como \{ isPlaying, progress, duration, isLoading, hasError \}.
        \item \textbf{Ações (actions):} Funções para controlar o player, como \{ handlePlay, handlePause, handleSeek, toggleMute \}.
    \end{itemize}
\end{itemize}

\subsubsection{Construção e Otimização de Performance}
A performance foi tratada como um requisito fundamental, especialmente em interações de alta frequência.

\textbf{Evento onProgress:} Este evento pode disparar múltiplas vezes por segundo, o que, se gerenciado ingenuamente com useState, causaria um excesso de re-renderização no React, sobrecarregando e resultando em uma UI "travada".

\textbf{Solução de Engenharia:}
\begin{itemize}
    \item \textbf{Manipulação Direta do DOM para UI:} A atualização visual da barra de progresso, que precisa ser fluida, foi desacoplada do estado do React. Utilizei o hook useRef para obter uma referência direta ao elemento da barra e atualizei seu estilo (transform: scaleX(...)) diretamente, bypassando o ciclo de renderização do React para essa animação específica.
    \item \textbf{Throttling para o Estado React:} O estado global do React (ex: o texto "01:32 / 05:00") ainda precisava ser atualizado, mas não com a mesma frequência. Implementei uma função de throttle (usando a biblioteca lodash, por exemplo) para garantir que o setState fosse chamado no máximo uma vez a cada 250ms, garantindo a atualização da informação sem sobrecarregar a aplicação.
    \item \textbf{Memoização:} Todos os sub-componentes da UI (ex: <PlayPauseButton />, <VolumeControl />) foram envolvidos em React.memo, e as funções de callback passadas como props foram estabilizadas com useCallback. Isso previne re-renderizações desnecessárias quando o estado pai muda, mas as props específicas daquele componente filho permanecem as mesmas.
\end{itemize}

\subsubsection{Testes Unitários do Hook useVideoPlayerState}
A qualidade e a robustez do componente foram garantidas através de testes unitários focados na lógica de estado, implementados com Jest e React Testing Library. A abordagem foi testar o custom hook em total isolamento, simulando os eventos da biblioteca de vídeo e verificando as transições de estado e as ações executadas, na Tabela \ref{tab:testes_video}, estão evidenciados os cenários de testes unitários abordados.

\begin{table}[h!]
    \centering
    \caption{Testes unitários na Implementação de Componente de Vídeo}
    \begin{tabular}{|p{0.45\textwidth}|p{0.45\textwidth}|}
        \hline
        \textbf{Objetivo} & \textbf{Verificação (Assert)} \\ \hline
        Garantir que o hook inicie no estado correto & O estado inicial isPlaying deve ser false, progress deve ser 0, isLoading deve ser true (ou false, dependendo da estratégia de autoplay). \\ \hline
        Validar a mudança de estado ao simular eventos de play e pause. & Ao simular o evento onPlay da biblioteca, o estado isPlaying deve se tornar true. Ao simular onPause, deve se tornar falso. \\ \hline
        Assegurar que o progresso e a duração do vídeo são refletidos no estado corretamente. & Ao simular um evento onProgress com \{ played: 0.5, loaded: 0.8 \}, o estado progress deve ser atualizado para 0.5. \\ \hline
        Verificar a transição para o estado ENDED. & Ao simular o evento onEnded, o estado isPlaying deve se tornar falso e um estado isEnded (se existir) deve ser true. \\ \hline
        Garantir que o hook lide corretamente com erros do player. & Ao simular um evento onError, o estado hasError deve se tornar true, e isLoading deve ser falso. \\ \hline
        Validar que a ação do usuário para buscar uma posição no vídeo é corretamente processada. & Ao chamar a ação handleSeek(0.75), o mock da função de seek da biblioteca de vídeo deve ser chamado com o argumento 0.75. \\ \hline
        Verificar se a ação de mutar/desmutar o áudio funciona. & Chamar toggleMute deve alterar o estado isMuted de false para true, e vice-versa. \\ \hline
    \end{tabular}
    \label{tab:testes_video}
    \source{Fonte: Do próprio autor}
\end{table}

\textbf{Tarefa ainda não finalizada:} Necessário a integração dentro das telas que farão o uso do componente. Esta próxima fase envolverá a passagem das props necessárias (como a URL do vídeo) e a conexão de callbacks (como onEnded) com a lógica de negócio da página que o contém.

\section{Estudos de Viabilidade Técnica para Funcionalidades de Vídeos com interatividades}

\subsection{Detalhamento da Execução}
Os estudos foram focados em gerar conhecimento técnico e reduzir riscos de negócio e implementação.

\subsection{Atuação em Tarefas (Visão de Engenharia de Software)}
\textbf{Tarefa: Análise de Trade-offs Arquiteturais para a Solução de vídeos interativos com hotspots.}

O estudo técnico envolve uma profunda análise de engenharia de sistemas, considerando custos, escalabilidade e manutenibilidade. Minha tarefa foi fornecer à liderança um relatório técnico para uma decisão informada.

\subsection{Execução}
\begin{itemize}
    \item \textbf{Análise de Requisitos Não-Funcionais:} Foi definido os critérios de avaliação: latência, escalabilidade , custo por espectador/hora e complexidade de manutenção.
    \item \textbf{Relatório Técnico e Recomendação:} Nele, apresentei a análise comparativa e recomendei a abordagem, alinhando a solução técnica com os objetivos de negócio (velocidade de lançamento), com um plano para reavaliar a estratégia se a escala crescesse exponencialmente.
\end{itemize}

\section{Implementação de Lógicas para Descontos (Período: 04/10 a 06/11)}

\subsection{Detalhamento da Execução}
Foco total em refatoração segura e melhoria da qualidade de um módulo crítico e legado.

\subsection{Atuação em Tarefas}
\textbf{Tarefa: Garantia de Qualidade e Prevenção de Regressão em Código Legado Crítico.}

Modificar um código complexo em um sistema legado e monolítico, sem testes e com alto impacto no negócio (cálculo de preços), sendo uma tarefa de risco, a prioridade absoluta foi garantir que o comportamento existente não fosse quebrado.

\subsection{Execução Baseada em Engenharia de Software}
\begin{itemize}
    \item \textbf{Engenharia Reversa e Testes de Caracterização:} Antes de alterar uma única linha, apliquei a técnica de Testes de Caracterização. Criei uma suíte de testes que não validava o que o código deveria fazer, mas o que ele realmente fazia, incluindo seus comportamentos inesperados. Isso criou uma rede de segurança que "caracterizou" o comportamento do sistema.
    \item \textbf{Refatoração Baseada em Padrões:} Com a segurança dos testes, liderei a refatoração. O código original era uma violação do Princípio Aberto/Fechado (para adicionar um novo desconto, era preciso modificar a função existente). Implementamos o Strategy Pattern, onde cada lógica de desconto se tornou uma classe separada com uma interface comum. A função principal agora apenas selecionava a "estratégia" correta, sem precisar conhecer seus detalhes.
    \item \textbf{Manutenibilidade:} A nova arquitetura tornou-se aberta para extensão (bastava criar uma nova classe de estratégia) e fechada para modificação, garantindo a manutenibilidade e a segurança para futuras alterações.
\end{itemize}

\textbf{Tarefa ainda não finalizada:} Necessário a criação de mais testes, a validação da tarefa em conjunto com o time de qualidade e a realização do deploy.

\section{Implementação de Funcionalidade para Imagens no Website (Período: 25/09 a 15/10)}

\subsection{Detalhamento da Execução}
Entregue de forma incremental, com foco inicial na funcionalidade e posterior na otimização de performance.

\subsection{Atuação}
\textbf{Tarefa: Otimização de Performance e Métricas de Core Web Vitals (CWV).}

Minha atuação como engenheiro foi focada em garantir que a funcionalidade não degradasse as métricas de performance (LCP, CLS), que impactam diretamente a experiência do usuário e o ranking no Google.

\subsection{Execução Baseada em Engenharia de Software}
\begin{itemize}
    \item \textbf{Análise e Diagnóstico:} Utilizei ferramentas de profiling (Lighthouse, Chrome DevTools) para medir o impacto da implementação inicial. O diagnóstico foi claro: o carregamento de múltiplas imagens de alta resolução estava atrasando o LCP (Largest Contentful Paint) e o carregamento tardio causava CLS (Cumulative Layout Shift).
    \item \textbf{Otimização Estrutural:} Apliquei uma abordagem multifacetada:
    \begin{itemize}
        \item Usei o componente <Image> do Next.js, que automaticamente serve imagens em formatos modernos (WebP) e tamanhos otimizados.
        \item Adicionei a prop priority na primeira imagem do carrossel. Isso é uma diretiva de engenharia que sinaliza ao navegador para priorizar o download deste recurso, melhorando diretamente o LCP.
        \item Para as demais imagens, implementei lazy loading, fazendo com que fossem carregadas apenas quando se aproximasse do viewport.
        \item Para resolver o CLS, especifiquei as dimensões exatas de cada imagem, para que o navegador pudesse reservar o espaço em tela antes do download ser concluído, evitando o "salto" no layout.
    \end{itemize}
    \item \textbf{Validação:} Após as otimizações, rodei novamente as ferramentas de profiling para validar que as métricas de CWV estavam dentro dos limites recomendados, tratando a performance como um requisito funcional do projeto.
    \item \textbf{Publicação de nova versão:} A versão foi publicada com as alterações, este deploy ocorreu no período da noite por conta dos riscos de serem realizados durante o dia.
    \item \textbf{Testes unitários realizados:} Através dos testes unitários, na Tabela \ref{tab:testes_imagens}, são evidenciados os critérios de assertividade da funcionalidade para garantir a integridade e a qualidade em ambiente local de desenvolvimento. Estes mesmos testes são executados antes do deploy da aplicação.
\end{itemize}

\begin{table}[h!]
    \centering
    \caption{Testes unitários na Implementação de Funcionalidade para Imagens no Website}
    \begin{tabular}{|p{0.45\textwidth}|p{0.45\textwidth}|}
        \hline
        \textbf{Objetivo} & \textbf{Verificação (Assert)} \\ \hline
        Validar a renderização inicial do componente. & Verificar se o número de elementos de imagem renderizados no DOM é igual ao tamanho do array passado na prop 'images'. \\ \hline
        Testar a interatividade do botão 'próximo'. & Simular um clique no botão de avançar e verificar se a função de callback 'onNextClick' foi chamada exatamente uma vez. \\ \hline
        Testar a interatividade do botão 'anterior'. & Simular um clique no botão de voltar e verificar se a função de callback 'onPrevClick' foi chamada exatamente uma vez. \\ \hline
        Validar o estado desabilitado do botão 'anterior' no início. & Renderizar o componente no estado inicial e verificar se o botão 'anterior' possui o atributo 'disabled'. \\ \hline
        Validar o estado desabilitado do botão 'próximo' no final. & Simular a navegação até o último slide e verificar se o botão 'próximo' possui o atributo 'disabled'. \\ \hline
        Testar a sincronização da UI dos indicadores de paginação. & Verificar se o indicador (dot) correspondente ao slide ativo possui uma classe CSS de 'active' e os outros não. \\ \hline
        Validar a renderização condicional baseada em props. & Renderizar o componente com a prop 'showArrows' como 'false' e verificar se os botões de navegação não existem no DOM. \\ \hline
    \end{tabular}
    \label{tab:testes_imagens}
    \source{Fonte: Do próprio autor}
\end{table}

\section{Atualização da versão do React Native (Período: 25/09 a 15/10)}
Uma atividade de alto risco focada em gerenciamento de débito técnico e configuração de ambiente.

\subsection{Atuação em Tarefas}
\textbf{Gerenciamento de Dependências Nativas Conflitantes e do Processo de Build.}

A complexidade de uma atualização do React Native está no código JavaScript, e também no ecossistema nativo subjacente. É uma tarefa de Engenharia de Configuração dos componentes nativos, atualização de bibliotecas do React-Native e correção de builds. Na Tabela \ref{tab:alteracoes_rn}, estão as alterações após a migração da versão anterior para a nova atualização da versão 0.81.

\subsection{Execução}
\begin{itemize}
    \item \textbf{Análise de Risco e Planejamento Estratégico:}
    \begin{itemize}
        \item \textbf{Análise de Breaking Changes:} Antes de iniciar, utilizei a ferramenta oficial react-native-upgrade-helper. Ela fornece um diff detalhado entre as versões, mostrando todas as alterações necessárias em arquivos de template nativos e de configuração, o que foi crucial para estimar o esforço e os pontos de risco.
        \item \textbf{Mapeamento de Dependências:} Criei um inventário de todas as bibliotecas de terceiros, verificando seus changelogs e issues no GitHub para confirmar a compatibilidade com a nova versão do React Native.
        \item \textbf{Plano de Execução:} Defini um plano em uma branch Git isolada (feature/upgrade-rn-0.xx), que incluía um plano de rollback claro: a qualquer momento, se um impedimento intransponível fosse encontrado, a branch seria descartada, garantindo zero impacto na main branch.
    \end{itemize}
    \item \textbf{Gerenciamento de Build Nativo:} A atualização quebrou as configurações do Gradle (Android) e do CocoaPods (iOS). Meu trabalho foi mergulhar nesses ecossistemas nativos, entender as breaking changes e aplicar as correções manualmente, o que exigiu conhecimento de android nativo.
    \item \textbf{Solução de Conflitos de Dependências:} O problema mais crítico foi uma biblioteca de SDK com a Google que dependia de uma versão antiga de uma biblioteca nativa do Android, enquanto a nova versão do React Native exigia uma mais recente. A solução de engenharia foi usar as ferramentas do react-native para realizar uma atualização pontual, através de um patch, e então aplicar esse ajuste no código da biblioteca da Google (usando patch-package) para torná-la compatível.
    \item \textbf{Validação:} Após os ajustes, garantindo que o processo automatizado estivesse validado antes de mesclar as mudanças para a branch principal. Isso garantiu a reprodutibilidade do build para toda a equipe e passagem para as próximas etapas de testes com o analista de qualidade.
    \item \textbf{Testes unitários:} Garantindo a qualidade de entrega, estão mapeados na Tabela \ref{tab:testes_rn}, os testes unitários que foram necessários adicionar após a migração da versão do react-native.
\end{itemize}

\begin{table}[h!]
    \centering
    \caption{Alterações após migração da versão do React-native}
    \begin{tabular}{|p{0.3\textwidth}|p{0.3\textwidth}|p{0.3\textwidth}|}
        \hline
        \textbf{Ponto de Alteração} & \textbf{Descrição Técnica da Mudança} & \textbf{Solução} \\ \hline
        Atualização do Android Gradle & A nova versão do React Native exige uma versão mais recenteo Gradle. Isso implicou em atualizar as dependências no build.gradle do projeto e a URL de distribuição no gradle-wrapper.properties. & A nova versão do AGP introduziu breaking changes na sintaxe do build.gradle, especialmente na forma como as buildFeatures e o namespace são declarados. Solução: Foi necessário a atualização do gradle \\ \hline
        Migração para TurboModules & Nossos módulos nativos customizados (escritos em Java) seguiam o padrão antigo (ReactContextBaseJavaModule). A Nova Arquitetura exige a migração para TurboModules. & Exigiu a reescrita de um dos módulos nativos. O código Java teve que ser adaptado para implementar as interfaces geradas pela especificação. Solução: Abordei a migração do módulo, garantindo o build do projeto após a migração do novo TurboModule. \\ \hline
        Requisitos do Ecossistema iOS (Xcode) & A versão 0.81 passou a exigir uma versão mais recente do Xcode (ex: Xcode 16) e do iOS SDK (ex: iOS 17). & Isso representou um desafio de infraestrutura. Foi necessário atualizar o ambiente de desenvolvimento de todos os membros da equipe. \\ \hline
        Depreciação de APIs Core & A nova versão removeu alguns componentes e APIs do core do React Native, movendo-os para pacotes da comunidade (ex: react-native-community). & Instalação da nova dependência (@react-native-clipboard/clipboard), garantindo que nenhuma ocorrência fosse esquecida. \\ \hline
    \end{tabular}
    \label{tab:alteracoes_rn}
    \source{Fonte: Do próprio autor}
\end{table}

\begin{table}[h!]
    \centering
    \caption{Testes unitários na Atualização da versão do React Native}
    \begin{tabular}{|p{0.3\textwidth}|p{0.3\textwidth}|p{0.3\textwidth}|}
        \hline
        \textbf{Tipo de Teste} & \textbf{Ferramentas/Método} & \textbf{Verificação (Assert)} \\ \hline
        Smoke Test (Manual) & Build de Teste (APK/IPA) & A aplicação abre sem crashar após a instalação limpa em um dispositivo físico com Android 14 e iOS 17. \\ \hline
        Regressão de Módulos Nativos & Teste Manual Focado & Verificar se a funcionalidade do módulo de integração com o SDK do Google (agora como TurboModule) continua operando corretamente. \\ \hline
        Regressão de Performance & Flipper, Perfetto, Xcode Instruments & Medir o tempo de inicialização (TTO) e o uso de memória em repouso. Comparar os resultados com a baseline da versão 0.76. \\ \hline
    \end{tabular}
    \label{tab:testes_rn}
    \source{Fonte: Do próprio autor}
\end{table}


\section{Migração de Arquitetura Bridge para JSI, Fabric e Turbo Module da biblioteca de (Período: 16/10 a 31/10)}

A tarefa consistiu na correção de uma breaking change crítica, introduzida após a atualização de versão do React Native. A mudança exigiu a refatoração de uma biblioteca interna, com modificações diretas no código nativo (Android/iOS) para adequação à Nova Arquitetura (JSI/Turbo Modules).

Para um diagnóstico preciso da falha, foi realizada uma pesquisa aprofundada em fontes técnicas, incluindo a documentação oficial (GitHub do React Native), consultas no Stack Overflow e o uso de ferramentas de assistência de código (Cursor AI).

O principal desafio metodológico foi o ciclo de feedback (build/test cycle) do projeto principal, que consumia aproximadamente 40 minutos por alteração. Esse overhead tornava a depuração Iterativa impraticável e arriscada.

Para mitigar esse risco e aumentar a agilidade, foi adotada a estratégia de desenvolvimento em ambiente isolado (Sandboxing). A correção foi implementada e validada em um aplicativo de demonstração ("sample app") minimalista, permitindo ciclos rápidos de desenvolvimento e depuração antes da integração.

O processo seguiu um pipeline de engenharia de software estruturado, desde o desenvolvimento local até a produção, conforme detalhado nas etapas abaixo.

\subsection{Desenho da solução}

A Figura \ref{fig:comunicacao_jsi} ilustra o processo de criação de um componente e as funções que possibilitam a comunicação entre o código React Native e os módulos nativos.

\begin{figure}[h!]
    \centering
    % \includegraphics[width=\textwidth]{figura1.png} % Placeholder para a imagem
    \caption{Comunicação JSI entre React-native e Nativo.}
    \label{fig:comunicacao_jsi}
    \source{Fonte: Do próprio autor.}
\end{figure}

\subsection{Desenvolvimento}
O desenvolvimento foi separado entre duas implementações: a Implementação A utilizando a arquitetura legada por bridge e a Implementação B com a nova arquitetura após a migração. O exemplo utilizado será um simples print da mensagem "Olá mundo".

\subsubsection{Implementação A - Arquitetura Legada}

A implementação abaixo utiliza a arquitetura legada, empregando o módulo NativeModules. Na camada JavaScript da aplicação, a comunicação com o módulo nativo é estabelecida por meio da importação de NativeModules do pacote react-native.

\begin{lstlisting}[language=JavaScript, caption={App.js (Lado JS)}]
import React from 'react';
import { Button, View, NativeModules } from 'react-native';

const { HelloWorldModule } = NativeModules;

const App = () => {
  const handlePressLegacy = async () => {
    try {
      const saudacao = await HelloWorldModule.getHelloWorld();
      
      console.log(saudacao); // Imprime "Olá mundo (Legado via Bridge)"
    
    } catch (e) {
      console.error(e);
    }
  };

  return (
    <View style={{ flex: 1, justifyContent: 'center' }}>
      <Button
        title="Testar Bridge Legada"
        onPress={handlePressLegacy}
      />
    </View>
  );
};

export default App;
\end{lstlisting}

No Código Nativo (Android - Java) foi realizada a implementação de um método chamado getHelloWorld para o retorno de uma string simples e um método getName para o retorno do nome do módulo nativo, sendo implementada da seguinte maneira:

\begin{lstlisting}[language=Java, caption={HelloWorldModule.java}]
// /android/app/src/main/java/.../HelloWorldModule.java
package com.meuapp;

import com.facebook.react.bridge.ReactApplicationContext;
import com.facebook.react.bridge.ReactContextBaseJavaModule;
import com.facebook.react.bridge.ReactMethod;
import com.facebook.react.bridge.Promise;

public class HelloWorldModule extends ReactContextBaseJavaModule {

    public HelloWorldModule(ReactApplicationContext context) {
        super(context);
    }

    @Override
    public String getName() {
        return "HelloWorldModule"; // Nome usado no JS
    }

    // Método exposto para o JS
    @ReactMethod
    public void getHelloWorld(Promise promise) {
        // A ponte é assíncrona, então usamos Promise
        promise.resolve("Olá mundo (Legado via Bridge)");
    }
}
\end{lstlisting}

No código nativo (iOS - Objective-C), análogo ao código em Java, a implementação foi realizada através de:

\begin{lstlisting}[language=C++, caption={HelloWorldModule (Objective-C)}]
// HelloWorldModule.h
#import <React/RCTBridgeModule.h>
// O módulo deve implementar o protocolo <RCTBridgeModule>
@interface HelloWorldModule : NSObject <RCTBridgeModule>
@end

// HelloWorldModule.m
#import "HelloWorldModule.h"
@implementation HelloWorldModule
// Expõe o módulo para o JS com o nome "HelloWorldModule"
RCT_EXPORT_MODULE(HelloWorldModule);
// Expõe o método getHelloWorld para o JS
// A chamada é assíncrona e usa Promises
RCT_EXPORT_METHOD(getHelloWorld: (RCTPromiseResolveBlock)resolve
                  rejecter: (RCTPromiseRejectBlock)reject)  {
    NSString *saudacao = @"Olá mundo (Legado via Bridge)";
    if (saudacao) {
        resolve(saudacao);
    } else {
        reject(@"error", @"Não foi possível gerar a saudação", nil);
    }
}
@end
\end{lstlisting}

\textbf{Funcionamento da Bridge:}
\begin{itemize}
    \item \textbf{Chamada JS:} O JavaScript chamava HelloWorldModule.getHelloWorld()
    \item \textbf{Serialização:} O React Native serializava o nome do módulo (HelloWorldModule) e o nome do método (getHelloWorld) em uma mensagem JSON
    \item \textbf{Comunicação:} A mensagem JSON era enviada da thread JavaScript para a thread Nativa, atravessando a Bridge
    \item \textbf{Processamento:} A thread Nativa desserializava a mensagem JSON, encontrava a classe HelloWorldModule e chamava o método correspondente
    \item \textbf{Retorno:} O resultado era serializado e retornado via callback/promise para o JavaScript
\end{itemize}

\subsubsection{Implementação B - Nova arquitetura}
No exemplo de implementação do logMessage com TurboModule, a migração demandou um arquivo de especificação (Spec) em TypeScript para definição da API, o qual serviu de base para geração do código boilerplate nativo.

\textbf{Definição da API (Spec - TypeScript)}

Um arquivo de especificação define o contrato da API de forma explícita na camada javascript, da seguinte maneira:

\begin{lstlisting}[language=TypeScript, caption={NativeHelloWorld.ts (Spec)}]
// /js/NativeHelloWorld.ts (A "Spec")
import type { TurboModule } from 'react-native/Libraries/TurboModule/TurboModule';
import { TurboModuleRegistry } from 'react-native';

// 1. Definimos a interface
export interface Spec extends TurboModule {
  // 2. Definimos um método SÍNCRONO
  getHelloWorld(): string; 
}

// 3. Registramos o módulo (o nome 'RTNHelloWorld' é o nome oficial do componente)
export default TurboModuleRegistry.getEnforcing<Spec>(
  'RTNHelloWorld',
);
\end{lstlisting}

\textbf{Implementação Nativa (Android - Kotlin)}

Na implementação nativa, foi utilizada uma interface gerada que permitiu o acesso direto ao runtime JavaScript.

\begin{lstlisting}[language=Java, caption={RTNHelloWorldModule.java}]
// /android/app/src/main/java/.../RTNHelloWorldModule.java
package com.meuapp;

import com.facebook.react.bridge.ReactApplicationContext;
import com.facebook.react.bridge.ReactMethod;
import androidx.annotation.NonNull;

// 1. O Codegen gera 'NativeHelloWorldSpec'
import com.facebook.fbreact.specs.NativeHelloWorldSpec; 

// 2. Implementamos a Spec gerada
public class RTNHelloWorldModule extends NativeHelloWorldSpec {

    public static final String NAME = "RTNHelloWorld";

    public RTNHelloWorldModule(ReactApplicationContext context) {
        super(context);
    }

    @Override
    @NonNull
    public String getName() {
        return NAME;
    }

    // 3. Implementação SÍNCRONA!
    //    Sem 'Promise', o retorno é direto.
    @Override
    public String getHelloWorld() {
        return "Olá mundo (Nova Arquitetura via JSI)";
    }
}
\end{lstlisting}

\textbf{Código em JavaScript}

A interface de uso no JavaScript conservou estrutura similar, mas passou a acessar diretamente o módulo C++ por meio da JSI.

\begin{lstlisting}[language=JavaScript, caption={App.js (Lado JS com Nova Arquitetura)}]
// App.js (Lado JS)
import React from 'react';
import { Button, View } from 'react-native';

// 1. Importa diretamente o módulo (conforme a Spec)
import RTNHelloWorld from './js/NativeHelloWorld'; 

const App = () => {

  const handlePressNewArch = () => {
    // 2. A chamada é SÍNCRONA!
    //    - Sem async/await, sem .then()
    //    - A JSI permite a chamada direta
    const saudacao = RTNHelloWorld.getHelloWorld();
    
    console.log(saudacao); // Imprime "Olá mundo (Nova Arquitetura via JSI)"
  };

  return (
    <View style={{ flex: 1, justifyContent: 'center' }}>
      <Button
        title="Testar Nova Arquitetura"
        onPress={handlePressNewArch}
      />
    </View>
  );
};

export default App;
\end{lstlisting}

\textbf{Implementação Nativa (iOS - Objective-C)}

Para a plataforma iOS, implementou-se o código nativo em Objective-C.

\begin{lstlisting}[language=C++, caption={RTNHelloWorld (Objective-C)}]
// RTNHelloWorld.h
// 1. Importa a Spec gerada pelo Codegen (baseada no seu .ts)
#import <RTNHelloWorld/RTNHelloWorldSpec.h> 

// 2. O nome da classe deve corresponder ao que o JS espera (RTNHelloWorld)
// 3. A classe implementa o protocolo da Spec gerada
@interface RTNHelloWorld : NSObject <NativeHelloWorldSpec> 
@end

// RTNHelloWorld.mm
#import "RTNHelloWorld.h"

@implementation RTNHelloWorld

// 1. Exporta o módulo (o nome é o da classe)
RCT_EXPORT_MODULE(RTNHelloWorld)

// 2. Esta é a implementação SÍNCRONA da Spec
- (NSString *)getHelloWorld
{
    // O retorno é direto, sem Promises, sem Bridge
    return @"Olá mundo (Nova Arquitetura via JSI)";
}

/*
   NOTA: O Codegen e o template de TurboModule cuidam de
   todo o "encanamento" C++ (JSI) para que este método
   possa ser chamado diretamente pelo JavaScript.
*/

@end
\end{lstlisting}

A nova arquitetura do React Native substituiu o sistema antigo de comunicação entre JavaScript e código nativo. No lugar da bridge assíncrona que funcionava como um intermediário lento, neste novo formato tem-se uma conexão direta e síncrona. Esta arquitetura tem três componentes principais: primeiro, o JSI que permite ao JavaScript chamar funções nativas diretamente, sem serialização de dados. Segundo, o TurboModules que gerencia essa comunicação de forma eficiente. E terceiro, o Codegen que automaticamente gerou todo o código de conexão necessário. A vantagem neste formato é que tudo acontece de forma síncrona. Não há espera por callbacks ou promises - as chamadas são instantâneas. Isso tornou a aplicação significativamente mais rápida e responsiva.

\subsection{Execução do módulo em um projeto de demonstração}
Esta etapa foi a aplicação direta da estratégia de mitigação de risco para contornar o build de 40 minutos do projeto principal.

\textbf{Descrição:} Um aplicativo React Native mínimo ("sample app") foi criado e configurado. O objetivo deste ambiente foi isolar totalmente a biblioteca do restante do ecossistema complexo do projeto principal.

\textbf{Execução:} A biblioteca modificada foi vinculada localmente a este projeto de demonstração (utilizando yarn link ou referência de diretório no package.json).

\textbf{Validação:} Esta abordagem permitiu ciclos de feedback quase instantâneos. Foi possível validar rapidamente:
\begin{itemize}
    \item A correta compilação do código nativo (Kotlin/Swift).
    \item A vinculação bem-sucedida do TurboModule através da JSI.
    \item A execução da funcionalidade central, confirmando que a comunicação entre a camada JavaScript e as novas implementações nativas estava ocorrendo conforme a especificação da API.
\end{itemize}

\subsection{Publicação de nova versão da biblioteca}
Após a validação funcional no ambiente isolado, a biblioteca precisou ser disponibilizada para consumo pelo projeto principal.

\textbf{Descrição:} O processo de empacotamento e versionamento da biblioteca para distribuição interna.

\textbf{Execução:} O processo seguiu o Versionamento Semântico (SemVer), onde a versão do pacote foi incrementada.

\textbf{Processo:} O build de distribuição da biblioteca foi gerado (compilando o TypeScript e preparando os artefatos nativos via react-native-builder-bob). O pacote foi então publicado em um registro de pacotes privados (NPM privado), garantindo que a nova versão estivesse acessível para o pipeline de CI/CD do aplicativo principal.

\subsection{Testes em ambiente de homologação (QA)}
Este foi o primeiro nível de integração real, onde a correção encontrou o ecossistema completo do aplicativo.

\textbf{Descrição:} O ambiente de homologação (QA) simula a infraestrutura de produção, mas com dados controlados e isolados, destinado à validação pela equipe de Qualidade.

\textbf{Execução:} O aplicativo principal foi atualizado para consumir a nova versão da biblioteca recém-publicada. Um novo build do aplicativo foi gerado e distribuído para a equipe de QA.

\textbf{Validação:} A equipe de QA executou um plano de testes de regressão completo, focado não apenas na funcionalidade corrigida, mas também em áreas adjacentes. O objetivo foi garantir que (A) a correção funcionava conforme o esperado dentro do app complexo e (B) a atualização não introduziu efeitos colaterais (regressões) em outras partes do sistema.

\subsection{Testes em ambientes pré-produtivos (UAT/Staging)}
Esta etapa representou o "ensaio geral" antes da produção.

\textbf{Descrição:} O ambiente de Pré-Produção (ou Staging) é um espelho fiel do ambiente de Produção, utilizando a mesma infraestrutura e, idealmente, uma réplica recente (e anonimizada) do banco de dados produtivo.

\textbf{Execução:} O build aprovado em Homologação (QA) foi promovido para este ambiente.

\textbf{Validação:} Aqui ocorreu a validação final de aceitação (UAT - User Acceptance Testing) por parte dos stakeholders ou Product Owners. Além dos testes funcionais, foram avaliados aspectos não-funcionais, como a performance da comunicação JSI sob carga e a interação com serviços de backend reais. Esta foi a última verificação de segurança antes da exposição aos usuários finais.

\subsection{Deploy para ambiente produtivo}
A etapa final do pipeline, disponibilizando a correção para todos os usuários.

\textbf{Descrição:} O processo de lançamento da versão do aplicativo contendo a biblioteca atualizada para o público geral.

\textbf{Execução:} Seguindo as melhores práticas de Entrega Contínua (CD), o deploy foi realizado através de um lançamento em fases (Staged Rollout).

\textbf{Metodologia:} A nova versão foi liberada inicialmente para uma pequena porcentagem de usuários (ex: 1\%, depois 5\%, 10\%). Durante esse processo, a equipe monitorou ativamente os dashboards de performance e estabilidade (APM, Crashlytics) para identificar qualquer anomalia ou erro em tempo real. Com a confirmação da estabilidade, a distribuição foi gradualmente aumentada até atingir 100\% da base de usuários.

\section{Migração do react-native 0.71.5 para a 0.81.4 (Período: 16/10 a 31/10)}

O módulo nativo, empacotado originalmente com o react-native-builder-bob, tornou-se obsoleto e causava erro de renderização dos componentes de layout após o upgrade do React Native. Utilizando o boilerplate e as ferramentas de linking do react-native-builder-bob, foram reconfigurados os wrappers nativos. Este processo exigiu a criação de um arquivo de especificação TypeScript (.d.ts), que define formalmente a interface do módulo, garantindo a tipagem estática. Em seguida, foi adaptado os entry points nativos (em Kotlin para Android e Swift para iOS) para implementar a interface gerada pelo JSI, o que assegurou a comunicação síncrona, tipada e de alta performance entre o JavaScript e o código nativo. O upgrade do React Native da versão 0.71.5 para a 0.81.4 envolveu diversas mudanças estruturais, de dependências e de configuração. Abaixo está um resumo das principais alterações realizadas, conforme o relatório do Upgrade Helper.

\subsection{Desenvolvimento}

\begin{table}[h!]
    \centering
    \caption{Mudanças de Versão e Dependências}
    \begin{tabular}{|l|l|l|}
    \hline
    \textbf{Item} & \textbf{Versão Anterior (0.71.5)} & \textbf{Versão Atualizada (0.81.4)} \\ \hline
    React Native & 0.71.5 & 0.81.4 \\ \hline
    React & 18.2.0 & 19.1.0 \\ \hline
    Node.js (Engine) & Não especificado & >=20 (Mínimo exigido) \\ \hline
    TypeScript & 4.8.4 & \textasciicircum5.8.3 \\ \hline
    react-test-renderer & 18.2.0 & 19.1.0 \\ \hline
    Jest & \textasciicircum29.2.1 & \textasciicircum29.6.3 \\ \hline
    Prettier & \textasciicircum2.4.1 & 2.8.8 \\ \hline
    \end{tabular}
    \source{Fonte: Do próprio autor.}
\end{table}

Houve também a adição de novas dependências de desenvolvimento, como:
\begin{itemize}
    \item Inclusão dos pacotes CLI da comunidade na versão 20.0.0 (@react-native-community/cli, cli-platform-android, cli-platform-ios).
    \item Inclusão de novos presets e configurações específicas do React Native: @react-native/babel-preset, @react-native/eslint-config, @react-native/metro-config, e @react-native/typescript-config, todos na versão 0.81.48.
    \item Remoção de pacotes antigos como @react-native-community/eslint-config e @tsconfig/react-native.
    \item O arquivo .node-version foi deletado.
\end{itemize}

\subsubsection{Destaques das Mudanças de Versões Intermediárias}
O upgrade passou por versões que introduziram mudanças significativas:

\begin{table}[h!]
    \centering
    \caption{Principais mudanças na atualização do RN.}
    \begin{tabular}{|l|p{10cm}|}
    \hline
    \textbf{Versão} & \textbf{Principais Mudanças} \\ \hline
    0.77 & Alterar o template do AppDelegate de Obj-C++ para Swift. Se for mantido o AppDelegate.mm, é necessário adicionar a linha RCTAppDependencyProvider. \\ \hline
    0.74 & Inclui Yoga 3.0 e o modo Bridgeless por padrão na Nova Arquitetura. O SDK Mínimo do Android agora é 23 (Android 6.0). Adiciona suporte a Yarn 3, e remove PropTypes previamente depreciados. \\ \hline
    0.73 & Inclui um processo atualizado para o Manifesto de Privacidade do iOS, agora obrigatório pela Apple. \\ \hline
    0.72 & Inclui uma nova configuração do Metro e um processo atualizado para o Manifesto de Privacidade do iOS. \\ \hline
    \end{tabular}
    \source{Fonte: Do próprio autor.}
\end{table}

\subsubsection{Alterações nos Arquivos de Configuração}
\begin{description}
    \item[.eslintrc.js:] O extend do ESLint foi alterado de @react-native-community para @react-native.
    \item[.gitignore:] Foram adicionadas regras para ignorar arquivos relacionados ao Kotlin (.kotlin/). Foi atualizado a regra para ignorar dependências do CocoaPods de /ios/Pods/ para **/Pods/. Foram adicionadas regras para ignorar arquivos temporários do Metro (.metro-health-check*) e diretórios do Yarn v3 (.yarn/*, .yarn/plugins, .yarn/versions, etc.).
    \item[.prettierrc.js:] Foi realizada a remoção de duas regras de formatação: bracketSameLine: true e bracketSpacing: false.
    \item[Gemfile (iOS):] Foi realizado o aumento do requisito mínimo do Ruby para >=2.6.18. Foi realizada a atualização e exclusão de versões problemáticas do cocoapods e activesupport. Foi realizado a adição de bibliotecas da Ruby Standard Library removidas no Ruby 3.4.0, como bigdecimal, logger, benchmark e mutex\_m18.
\end{description}

\subsubsection{Refatoração do Arquivo App.tsx (Tela Inicial)}
A estrutura da tela inicial padrão (App.tsx) foi completamente refatorada, migrando de uma implementação manual dos componentes do boilerplate para o uso de um novo componente dedicado:
\begin{itemize}
    \item O código foi migrado para usar SafeAreaProvider e o hook useSafeAreaInsets do pacote react-native-safe-area-contex.
    \item O conteúdo principal da tela inicial (seções "Step One", "See Your Changes", etc.) foi substituído por um único componente: <NewAppScreen />, importado de @react-native/new-app-screen.
    \item O objeto styles padrão foi simplificado, mantendo apenas o estilo de container.
\end{itemize}

\subsubsection{Ferramenta de Upgrade}
Para a gestão de dependências neste upgrade, foi utilizado a ferramenta @rnx-kit/align-deps da Microsoft, que automatiza o alinhamento de versões de pacotes compatíveis com a versão específica do React Native. O comando utilizado foi:

\begin{lstlisting}[language=bash]
npx @rnx-kit/align-deps --requirements react-native@0.81.4
\end{lstlisting}

\subsection{Publicação de nova versão da biblioteca}
Após a validação no ambiente isolado, o módulo reconfigurado foi versionado (seguindo o Versionamento Semântico) e empacotado, tornando a nova versão corrigida disponível para ser consumida pelo pipeline de CI/CD do aplicativo principal.

\subsection{Testes em ambiente de homologação (QA)}
Nesta etapa, o aplicativo principal (agora migrado para a versão 0.81.4 do React Native) teve sua dependência atualizada para a nova versão da biblioteca. Um build foi gerado e disponibilizado para a equipe de QA. O escopo dos testes foi duplo:
\begin{itemize}
    \item \textbf{Teste de Correção:} Validar especificamente que o erro de renderização do componente de layout foi resolvido.
    \item \textbf{Teste de Regressão do Upgrade:} Executar um plano de testes completo para garantir que as vastas mudanças estruturais do upgrade (como Yoga 3.0, novo AppDelegate em Swift, mudanças no App.tsx e a remoção de PropTypes) não introduziram novos bugs e regressões funcionais.
\end{itemize}

\subsection{Testes em ambientes pré-produtivos (pré-produção)}
O build aprovado pela equipe de QA foi promovido para o ambiente de Pré-Produção. Este ambiente espelha a infraestrutura de produção, incluindo a conexão com backends e serviços reais. O objetivo foi validar o comportamento do aplicativo inteiramente atualizado (RN 0.81.4 e o módulo JSI corrigido) sob condições reais. Verificou-se as funcionalidades após a troca da comunicação JSI, a estabilidade do app após as mudanças de dependências e a compatibilidade com os sistemas nativos Android e IOS.

\subsection{Deploy para ambiente produtivo}
Dada a magnitude da atualização (uma migração de versão maior do React Native e a introdução de um módulo JSI refatorado), o deploy foi realizado com cautela. Utilizou-se a estratégia de Staged Rollout (lançamento em fases) através das lojas (App Store e Play Store). A nova versão foi liberada inicialmente para um pequeno percentual da base de usuários. Durante esse período, as ferramentas de monitoramento de performance (APM) e crash reporting (Crashlytics) foram acompanhadas ativamente para validar a estabilidade do upgrade e da nova implementação JSI em cenários de uso real, antes da liberação para 100\% dos usuários.

\section{Atualização Crítica do Google SDK (v10 para v12) (Período: 16/10 a 31/10)}

A atualização da dependência do Google SDK foi obrigatória por questões de segurança e compatibilidade, mas quebrou a funcionalidade do módulo nativo devido a breaking changes de API e conflitos de dependência. A solução foi um ajuste fino e especialista nas dependências nativas:

\begin{itemize}
    \item \textbf{Android (Gradle):} Realizei a atualização explícita do build.gradle do módulo para o Google SDK v12, aplicando, quando necessário, estratégias de Gradle Resolution Strategy para forçar a nova versão e garantir a coexistência com o restante do projeto. Finalizei com um clean build para resolver eventuais conflitos de recursos.
    \item \textbf{iOS (CocoaPods/SPM):} O Podfile do módulo foi revisado e foi garantido que a nova versão do SDK fosse vinculada corretamente, e que quaisquer breaking changes de API do Google SDK fossem absorvidas e adaptadas no código nativo do módulo (escrito em Swift), mantendo a estabilidade da aplicação iOS.
\end{itemize}

\subsection{Análise de Impacto e Planejamento}
A etapa inicial consistiu na análise de impacto da atualização obrigatória do Google SDK (v10 para v12). O objetivo foi mapear as breaking changes de API e os conflitos de dependência que quebravam o módulo nativo. O planejamento definiu uma abordagem de intervenção, onde foram realizados:
\begin{itemize}
    \item \textbf{Android:} Foi utilizado a propriedade Gradle Resolution Strategy para forçar a nova versão (v12) e garantir a coexistência com outras dependências do projeto.
    \item \textbf{iOS:} Mapeamento das APIs depreciadas e preparação da refatoração do código nativo em Swift, além de atualizar o Podfile.
\end{itemize}

\subsection{Desenvolvimento e Correção Nativa (Multiplataforma)}
Esta foi a etapa de execução técnica da correção, dividida por plataforma:
\begin{itemize}
    \item \textbf{Android (Gradle):} O arquivo build.gradle do módulo foi modificado para atualizar explicitamente a dependência. A resolutionStrategy foi aplicada no build.gradle do projeto host para resolver conflitos, forçando o Google SDK v12 em todo o dependency graph. A execução de um clean build (limpeza de cache) foi essencial para garantir que o Gradle utilizasse apenas os novos artefatos.
    \item \textbf{iOS (Swift/CocoaPods):} O Podfile foi atualizado para referenciar a nova versão do Google SDK. Após a instalação dos pods, o projeto foi aberto no Xcode e o código nativo em Swift, que interagia com o SDK, foi adaptado manualmente para as novas assinaturas de método e APIs (absorvendo as breaking changes).
\end{itemize}

\subsection{Execução do módulo em um projeto de demonstração}
Devido às dificuldades no processo de build do projeto principal, a validação da crítica mudança na dependência nativa foi realizada em um aplicativo de demonstração. Esta abordagem foi adotada para agilizar o debugging, contornando o build lento do projeto principal. No ambiente isolado do sample app, foi possível:
\begin{itemize}
    \item Vincular o módulo modificado localmente.
    \item Verificar que a funcionalidade anteriormente comprometida do módulo foi restaurada.
\end{itemize}

\subsection{Publicação de nova versão da biblioteca}
Com a correção validada no "sample app", a biblioteca foi preparada para distribuição interna. O versionamento semântico foi aplicado (ex: incremento de patch ou minor), e o pacote foi publicado no registro privado (Artifactory/NPM). Esta nova versão continha as alterações de configuração nativa (build.gradle e Podfile) e o código Swift adaptado, pronta para ser consumida pelo aplicativo principal.

\subsection{Testes em ambiente de homologação (QA)}
A nova versão da biblioteca foi atualizada no aplicativo principal e um build foi gerado para o ambiente de QA. A equipe de qualidade executou um plano de testes de regressão focado:
\begin{itemize}
    \item \textbf{Validação da Correção:} Garantir que a funcionalidade do módulo estava operacional.
    \item \textbf{Validação de Coexistência (Risco do Gradle):} O teste mais crítico foi verificar se a estratégia de forçar o Google SDK v12 não introduziu efeitos colaterais ou quebrou outros módulos que pudessem depender de sub-versões diferentes dos serviços do Google.
\end{itemize}
Ambos os dois testes acima foram validados e passados com sucesso.

\subsection{Testes em ambientes pré-produtivos (Staging)}
O build aprovado em QA foi promovido ao ambiente de Staging, um espelho do ambiente de produção. Esta etapa validou a interação da aplicação com os serviços reais do Google (autenticação, mapas, etc., dependendo do SDK) utilizando as novas APIs v12. O objetivo foi garantir que a atualização de segurança e compatibilidade funcionava corretamente em um cenário de uso real antes da exposição ao cliente.

\subsection{Deploy para ambiente produtivo}
Dada a natureza da mudança (atualização de segurança e correção de breaking changes), o deploy foi tratado com monitoramento intensivo. Foi utilizado um Staged Rollout (lançamento em fases) via Play Store e App Store. A versão foi liberada inicialmente para 1\% dos usuários, com acompanhamento em tempo real das ferramentas de crash reporting (Crashlytics). Após a confirmação da estabilidade (ausência de novos crashes nativos relacionados ao Google SDK), a distribuição foi escalonada para 100\%.

\section{Desenvolvimento de um MVP de um Sistema Web (Período: início em 16/10, tarefa ainda em andamento)}

\subsection{Detalhamento da Execução}
O MVP está sendo construído e tem previsão de acabar ao longo do final das atividades previstas na disciplina.

\subsection{Design e Implementação do API Gateway e do Contrato de Comunicação Inter-serviços}
A simples criação de endpoints individuais em cada micro serviço levaria a um acoplamento forte com o front-end e a uma complexidade de gerenciamento. Aplicando os princípios de Design de Software, foi implementada uma API Gateway para desacoplamento entre regras de negócios da aplicação.

Para demonstrar a aplicação técnica, a solução foi implementada utilizando Node.js com o framework Express, devido à sua leveza e eficiência em operações de I/O.

\subsubsection{Contrato OpenAPI como Fonte da Verdade}
O primeiro passo foi definir o contrato para o endpoint público no Gateway. Foi utilizada a especificação OpenAPI (Swagger) para documentar o formato de resposta esperado pelo front-end.

\textbf{Exemplo de Caso de Uso:} Obter detalhes de um Produto, que exige dados do Serviço de Produtos e do Serviço de Avaliações.

\begin{lstlisting}[language=yaml, caption={Trecho da Especificação OpenAPI (Gateway Público)}]
# Trecho da Especificação OpenAPI (Gateway Público)
paths:
  /v1/produtos/{id}:
    get:
      summary: Retorna detalhes de um produto com suas avaliações
      responses:
        '200':
          content:
            application/json:
              schema:
                type: object
                properties:
                  produtoId: { type: integer }
                  nome: { type: string }
                  preco: { type: number }
                  # Propriedade agregada:
                  mediaAvaliacoes: { type: number, description: "Média calculada pelo Gateway" }
                  totalAvaliacoes: { type: integer }
\end{lstlisting}

\subsubsection{Orquestração e Agregação no Código do Gateway}
O código do endpoint no Gateway demonstra a orquestração paralela dos serviços internos.

\begin{lstlisting}[language=JavaScript, caption={Exemplo de código no API Gateway}]
// Exemplo de código no API Gateway (Node.js com Axios para chamadas HTTP)

const express = require('express');
const axios = require('axios');
const router = express.Router();

const URL_PRODUTOS = 'http://servico-produtos:3001/api';
const URL_AVALIACOES = 'http://servico-avaliacoes:3002/api';

// GET /v1/produtos/:id
router.get('/v1/produtos/:id', async (req, res) => {
    const produtoId = req.params.id;
    
    // 1. Centralização da Autenticação (Cross-Cutting Concern)
    // Se a rota exigisse, a validação do token JWT ocorreria aqui.

    try {
        // 2. Orquestração Paralela: Chamadas simultâneas aos microsserviços internos
        const [produtoResponse, avaliacoesResponse] = await Promise.all([
            axios.get(`${URL_PRODUTOS}/produtos/${produtoId}`),
            axios.get(`${URL_AVALIACOES}/avaliacoes/produto/${produtoId}`)
        ]);

        const produto = produtoResponse.data;
        const avaliacoes = avaliacoesResponse.data;

        // 3. Agregação e Transformação de Dados
        const media = avaliacoes.length 
                      ? avaliacoes.reduce((soma, a) => soma + a.nota, 0) / avaliacoes.length 
                      : 0;

        // 4. Retorno ao Cliente (Aderente ao Contrato OpenAPI)
        const respostaGateway = {
            produtoId: produto.id,
            nome: produto.nome,
            preco: produto.valor,
            mediaAvaliacoes: parseFloat(media.toFixed(1)),
            totalAvaliacoes: avaliacoes.length
        };

        return res.status(200).json(respostaGateway);

    } catch (error) {
        // Tratamento centralizado de erros de comunicação inter-serviços
        console.error('Erro na orquestração:', error.message);
        return res.status(500).json({ erro: 'Falha ao processar a requisição.' });
    }
});
module.exports = router;
\end{lstlisting}

\subsection{Publicação de nova versão}
Atualmente a atuação está na publicação de uma nova versão e encontra-se em andamento.
As próximas etapas previstas são:
\begin{enumerate}
    \item Publicação de uma versão
    \item Testes em ambiente de homologação (QA)
    \item Testes em ambientes pré-produtivos (pré-produção)
    \item Deploy para ambiente produtivo
\end{enumerate}


\section{Estudo de Agentes Conversacionais: Fluxos Determinísticos (Período: 01/11 a 15/11)}

\subsection{Contextualização da tarefa}
Esta tarefa teve como objetivo um estudo para compreender os fundamentos de \textit{Conversational Agents}, focando especificamente em fluxos deterministicos. Os estudos envolveu o aprendizado de como estruturar conversas rígidas e controladas usando FSM.

\subsection{Objetivos de Aprendizado} Desenvolver competências em design de fluxos determinísticos (\textit{Flows}) para processos que exigem alta confiabilidade e auditabilidade.

O estudo focou em compreender como garantir que cada etapa de um fluxo conversacional seja validada, especialmente em contextos críticos como agendamento médico, onde não pode haver margem para erro ou ambiguidade.

\subsection{Fundamentação Teórica e Prática}

\subsubsection{Pesquisa sobre FSM}
A primeira etapa do estudo consistiu em pesquisar e compreender o conceito de FSM aplicado a chatbots. Estudei como traduzir processos de negócio em diagramas de estados, identificando:

\begin{itemize}
    \item \textbf{Estados (Pages):} Representam cada etapa do fluxo conversacional (ex: coleta de especialidade, data, horário).
    \item \textbf{Transições (Routes):} Condições que permitem a mudança de um estado para outro.
    \item \textbf{Validações:} Regras que garantem a integridade dos dados coletados antes de prosseguir.
\end{itemize}

\subsubsection{Estudo de NLP}
Aprofundei o conhecimento sobre como sistemas de NLP interpretam linguagem humana através de:

\textbf{Intents (Intenções):} Estudei como treinar o modelo para reconhecer diferentes formas de expressar a mesma intenção. Por exemplo, as frases abaixo representam a mesma intenção de agendamento:

\begin{lstlisting}[language=text, caption={Variações Linguísticas para Treinamento de Intent}]
``Quero marcar um cardiologista''
``Preciso de uma consulta para amanha''
``Tem vaga para doutor Paulo?''
``Gostaria de agendar um horario''
``Quero consulta com dermatologista''
\end{lstlisting}

\textbf{Entidades (Entities):} Compreendi como o sistema extrai informações específicas de frases em linguagem natural. Exemplo de análise:

Entrada do usuário: \textit{``Quero ir na unidade Centro na sexta-feira''}

Extração automática:
\begin{itemize}
    \item \texttt{@unidade}: ``Centro'' (entidade customizada)
    \item \texttt{@sys.date}: ``2024-11-29'' (entidade de sistema que converte expressões temporais)
\end{itemize}

\subsubsection{Aplicação Prática: Modelagem de Fluxo de Agendamento}
Com base nos estudos teóricos, modelei um fluxo conversacional completo aplicando o conceito de Slot Filling (preenchimento obrigatório de campos):

\begin{enumerate}
    \item \textbf{Estado ``Início'':} Captura a especialidade médica desejada.
    \item \textbf{Estado ``Coleta de Unidade'':} Permanece em loop até que o parâmetro \texttt{\$session.params.unidade} seja preenchido.
    \item \textbf{Estado ``Coleta de Data'':} Valida se a data informada é futura e está dentro do período permitido (30 dias).
    \item \textbf{Estado ``Coleta de Horário'':} Apresenta opções de horários disponíveis.
    \item \textbf{Estado ``Confirmação'':} Exibe resumo completo para validação do usuário.
    \item \textbf{Estado ``Finalização'':} Processa o agendamento e gera protocolo.
\end{enumerate}

\subsubsection{Estudo de Lógica Condicional em Fluxos Conversacionais}
Pesquisei como implementar rotas condicionais para controlar transições entre estados:

\begin{lstlisting}[language=text, caption={Exemplo de Lógica Condicional Estudada}]
SE $session.params.confirmacao = ``sim'' ENTAO
    Transitar para Estado ``Finalizacao''

SE $session.params.confirmacao = ``nao'' ENTAO
    Limpar todos os parametros
    Retornar para Estado ``Inicio''
\end{lstlisting}

\subsubsection{Análise de Estratégias de Tratamento de Erros}
Estudei técnicas para tornar sistemas conversacionais mais robustos:

\begin{itemize}
    \item \textbf{Validação Contextual:} Aprendi a implementar validações que consideram o contexto (ex: rejeitar datas passadas com mensagens explicativas).
    \item \textbf{Fallback Progressivo:} Pesquisei estratégias de degradação gradual, onde após múltiplas falhas de compreensão, o sistema oferece alternativas (ex: transferência para atendimento humano).
    \item \textbf{Princípio Fail-Safe:} Estudei a importância de confirmações explícitas em processos críticos, onde o usuário revisa todas as informações antes da execução final.
\end{itemize}

\subsubsection{Metodologia de Validação de Sistemas Conversacionais}
Como parte do estudo, pesquisei metodologias de teste para chatbots, resultando na elaboração de cenários de validação conforme a Tabela \ref{tab:testes_dialogflow}.

\begin{table}[h!]
    \centering
    \caption{Cenários de Teste Estudados para Validação de Chatbots}
    \begin{tabular}{p{0.45\textwidth}|p{0.45\textwidth}}
        \hline
        \textbf{Cenário de Teste} & \textbf{Comportamento Esperado (Aprendizado)} \\ \hline
        Fluxo completo com dados válidos & Compreender o caminho feliz e geração de confirmações. \\ \hline
        Tentativa de entrada inválida (data passada) & Estudar validações contextuais e mensagens de erro amigáveis. \\ \hline
        Usuário fornece informações incompletas & Analisar o comportamento de Slot Filling e persistência de estado. \\ \hline
        Usuário desiste durante o processo & Compreender estratégias de cancelamento e limpeza de sessão. \\ \hline
        Teste de fallback após múltiplas falhas & Estudar degradação gradual e transferência para canal alternativo. \\ \hline
        Integração com sistemas externos & Pesquisar comunicação via webhooks e tratamento de latência. \\ \hline
        Confirmação antes da ação crítica & Analisar padrões de UX para operações irreversíveis. \\ \hline
    \end{tabular}
    \label{tab:testes_dialogflow}
    \vspace{0.3em}
    \noindent{\small Fonte: Do próprio autor}
\end{table}

\subsubsection{Competências Desenvolvidas}
Ao final deste estudo, foram desenvolvidas as seguintes competências:
\begin{itemize}
    \item Modelagem de fluxos conversacionais usando FSM
    \item Compreensão de conceitos de NLP (intents, entities, context)
    \item Design de experiências conversacionais para processos críticos
    \item Estratégias de validação e tratamento de erros em chatbots
    \item Metodologias de teste para sistemas conversacionais
\end{itemize}

\section{Estudo de Arquiteturas Híbridas: Fluxos Parcialmente Generativos (Período: 10/11 a 20/11)}

\subsection{Contextualização da tarefa}
Esta tarefa focou em estudar a categoria de \textit{Partly Generative Flows}, que combina a segurança dos fluxos determinísticos com a flexibilidade da IA Generativa. A pesquisa focou no uso de \textit{Generators} para enriquecer interações sem perder o controle do fluxo principal.

\subsection{Objetivos de Aprendizado}
Desenvolver competências em design de arquiteturas híbridas que utilizam \textit{Generators} e \textit{Generative Fallback} estrategicamente.

O estudo explorou como tomar decisões arquiteturais baseadas em análise de risco: operações financeiras exigem fluxos rígidos e auditáveis, enquanto consultas informativas podem aproveitar a flexibilidade da IA generativa.

\subsection{Fundamentação Teórica e Prática}

\subsubsection{Pesquisa sobre Paradigmas}
Estudei metodologias para classificar funcionalidades por nível de risco e determinar qual paradigma aplicar em cada caso:

\begin{itemize}
    \item \textbf{Paradigma Determinístico:} Aplicado tanto em operações de alto risco (PIX e Boletos, via \textit{Flows}) quanto em consultas de saldo de médio risco (via \textit{API} Direta).
    \item \textbf{Paradigma Híbrido:} Utilizado para funcionalidades consultivas de médio risco, como sugestões de investimento.
    \item \textbf{Paradigma Generativo:} Destinado a interações educacionais de baixo risco, como a explicação de termos financeiros.
\end{itemize}


\subsubsection{Competências Desenvolvidas}
Ao final deste estudo, foram desenvolvidas as seguintes competências:
\begin{itemize}
    \item Análise de risco aplicada a decisões arquiteturais
    \item Design de sistemas híbridos com segregação de responsabilidades
    \item Compreensão de fluxos determinísticos para operações críticas
    \item Aplicação de IA generativa em contextos de baixo risco
    \item Técnicas de prompt engineering para modelos de linguagem
    \item Estratégias de hand-off entre paradigmas distintos
\end{itemize}

\section{Estudo de Arquitetura Serverless com Google Cloud Platform (GCP) Cloud Functions (Período: 15/11 a 24/11)}

\subsection{Contextualização da Tarefa}
Esta tarefa focou em compreender o paradigma de computação serverless e sua aplicação em integrações de sistemas. A pesquisa envolveu o aprendizado de webhooks, Cloud Functions e padrões de integração entre sistemas conversacionais e APIs legadas.

\subsection{Objetivos de Aprendizado}
\textbf{Objetivo Principal:} Desenvolver competências em design e implementação de backends serverless para integração de sistemas heterogêneos.

O estudo explorou como webhooks atuam como ponte entre camadas conversacionais (Dialogflow) e APIs internas, executando lógica de negócio e consultas a bancos de dados.

\subsection{Fundamentação Teórica e Prática}

\subsubsection{Pesquisa sobre Padrão API Gateway}
Estudei o padrão arquitetural API Gateway e como Cloud Functions podem atuar como orquestradores:

\begin{itemize}
    \item \textbf{Entrada:} Requisição HTTP POST do Agente contendo os parâmetros da sessão.
    \item \textbf{Processamento:} Lógica de negócio e chamadas a APIs internas (REST).
    \item \textbf{Saída:} Resposta formatada no protocolo do Agente (JSON estruturado).
\end{itemize}

\subsubsection{Implementação do Webhook de Rastreamento}
Para o caso de uso de rastreamento de encomendas em um sistema de logística, implementei a Cloud Function conforme detalhado na Listagem \ref{lst:webhook} do Apêndice \ref{apendice:api}.


\subsubsection{Boas Práticas Implementadas}

\begin{itemize}
    \item \textbf{Validação de Entrada:} Todos os parâmetros recebidos do Dialogflow são validados antes do processamento.
    \item \textbf{Timeout e Retry:} Implementei timeout de 5 segundos nas chamadas HTTP para evitar que o usuário fique esperando indefinidamente.
    \item \textbf{Tratamento de Erros:} Erros são logados (para debugging) e mensagens amigáveis são retornadas ao usuário.
    \item \textbf{Segurança:} Credenciais de API são armazenadas como variáveis de ambiente (Secret Manager do GCP), nunca no código.
    \item \textbf{Logging:} Utilizei o Cloud Logging para rastreabilidade e debugging em produção.
\end{itemize}

\subsubsection{Deploy e Configuração no GCP}
O processo de deploy seguiu as melhores práticas de CI/CD:

\begin{enumerate}
    \item \textbf{Desenvolvimento Local:} Testes locais usando o Functions Framework.
    \item \textbf{Versionamento:} Código versionado no Git com tags semânticas.
    \item \textbf{Deploy:} Utilizei o comando \texttt{gcloud functions deploy} com configuração de memória (256MB) e timeout (60s).
    \item \textbf{Configuração de Trigger:} A função foi configurada para aceitar requisições HTTP autenticadas.
    \item \textbf{Integração com o Agente:} A URL da Cloud Function foi configurada no webhook do Conversational Agent.
\end{enumerate}

\subsubsection{Testes e Validação}
A validação foi realizada em múltiplas camadas, conforme evidenciado na Tabela \ref{tab:testes_webhook}.

\begin{table}[h!]
    \centering
    \caption{Testes de Validação do Webhook de Rastreamento}
    \begin{tabular}{p{0.45\textwidth}|p{0.45\textwidth}}
        \hline
        \textbf{Cenário de Teste} & \textbf{Verificação (Assert)} \\ \hline
        Código de rastreamento válido & A função deve retornar o status correto da API da transportadora. \\ \hline
        Código de rastreamento inválido & A função deve retornar mensagem de erro amigável sem falhar. \\ \hline
        API da transportadora indisponível & A função deve retornar mensagem de erro após timeout e logar o erro. \\ \hline
        Teste de carga (100 req/s) & A Cloud Function deve escalar automaticamente sem degradação de performance. \\ \hline
        Validação de segurança & Requisições sem token de autenticação devem ser rejeitadas (HTTP 401). \\ \hline
        Teste de latência & O tempo de resposta total (Dialogflow + Webhook + API) deve ser inferior a 3 segundos. \\ \hline
    \end{tabular}
    \label{tab:testes_webhook}
    \vspace{0.3em}
    \noindent{\small Fonte: Do próprio autor}
\end{table}

\subsubsection{Competências Desenvolvidas}
Ao final deste estudo, foram desenvolvidas as seguintes competências:
\begin{itemize}
    \item Compreensão do paradigma de computação serverless
    \item Design e implementação de webhooks para integração de sistemas
    \item Aplicação do padrão API Gateway em arquiteturas distribuídas
    \item Tratamento de erros e validação de entrada em APIs
    \item Configuração e deploy de Cloud Functions no GCP
    \item Boas práticas de segurança (Secret Manager, autenticação)
    \item Metodologias de teste para integrações assíncronas
\end{itemize}

\section{Estudo de RAG com Data Stores e Vertex AI (Período: 20/11 a 24/11)}

\subsection{Contextualização da tarefa}
Esta tarefa teve como o foco estudar sobre a implementação de Retrieval-Augmented Generation (RAG) utilizando a abstração de \textit{Data Stores} do ecossistema Conversational Agents. A pesquisa focou em como conectar agentes a fontes de dados não estruturadas (PDFs, sites) de forma nativa.

\subsection{Objetivos de Aprendizado}
\textbf{Objetivo Principal:} Desenvolver competências em configuração de \textit{Data Stores}, ingestão de documentos e integração com modelos generativos para respostas fundamentadas.

O estudo explorou o desafio de criar assistentes capazes de responder perguntas sobre documentos extensos (ex: manual de 50 páginas) de forma precisa e contextualizada, mantendo citações às fontes originais.

\subsection{Fundamentação Teórica e Prática}

\subsubsection{Pesquisa sobre Arquitetura RAG}
Estudei o padrão arquitetural RAG e suas etapas fundamentais:

\begin{enumerate}
    \item \textbf{Ingestão de Documentos:} Upload do PDF ``Normas da Empresa 2024'' para o Vertex AI Search.
    \item \textbf{Chunking e Indexação:} O documento é dividido em chunks semânticos e indexado com embeddings vetoriais.
    \item \textbf{Retrieval (Busca):} Quando o usuário faz uma pergunta, o sistema busca os chunks mais relevantes usando similaridade vetorial.
    \item \textbf{Augmentation (Aumento):} Os chunks recuperados são injetados no contexto do prompt do modelo generativo.
    \item \textbf{Generation (Geração):} O modelo (Gemini Pro) gera a resposta baseada no contexto recuperado.
\end{enumerate}


\section{Estudo de Protocolos e Kits de Desenvolvimento para Agentes: ADK e MCP (Período: 10/11 a 20/11)}

\subsection{Contextualização da Tarefa}
Esta tarefa focou em estudar duas tecnologias emergentes: o \textit{Agent Development Kit} (ADK) para construção de agentes complexos e o \textit{Model Context Protocol} (MCP) para padronização da conexão entre modelos de IA e fontes de dados externas.

\subsection{Objetivos de Aprendizado}
\textbf{Objetivo Principal:} Compreender e implementar padrões modernos de arquitetura de agentes, focando na interoperabilidade e na capacidade de uso de ferramentas (tool use).

O estudo buscou superar as limitações de integrações ad-hoc (como webhooks rígidos) explorando o protocolo MCP, que permite que assistentes de IA descubram e utilizem recursos de forma dinâmica e padronizada.

\subsection{Fundamentação Teórica e Prática}

\subsubsection{Pesquisa sobre Model Context Protocol (MCP)}
Estudei a especificação do protocolo MCP, que define uma arquitetura cliente-servidor para conectar IAs a sistemas:

\begin{itemize}
    \item \textbf{MCP Hosts:} Aplicações que ``consomem'' o contexto (ex: IDEs, Chatbots).
    \item \textbf{MCP Servers:} Serviços que expõem recursos, prompts e ferramentas.
    \item \textbf{MCP Clients:} O conector que mantém a sessão entre Host e Server.
\end{itemize}

A principal vantagem identificada foi a dissociação: um único servidor MCP (ex: ``Conector Google Drive'') pode ser reutilizado por múltiplos agentes sem reescrita de código.

Para fixar o aprendizado, desenvolvi um servidor MCP em TypeScript que expõe uma base de dados local de produtos para um agente (ver Listagem \ref{lst:mcp-server} no Apêndice \ref{apendice:api}).


\subsubsection{Estudo de Agent Development Kit (ADK)}
Explorei frameworks e kits de desenvolvimento (ADK) que facilitam a orquestração de agentes. O foco foi entender como esses kits abstraem a complexidade de:

\begin{itemize}
    \item \textbf{Gerenciamento de Memória:} Como o agente ``lembra'' de interações passadas.
    \item \textbf{Planejamento (Planning):} Capacidade do agente de quebrar uma tarefa complexa (``Planejar viagem'') em sub-tarefas (``Ver voos'', ``Reservar hotel'').
    \item \textbf{Execução de Ferramentas:} O uso autônomo de ferramentas expostas via MCP ou APIs.
\end{itemize}

\subsubsection{Competências Desenvolvidas}
Ao final desta tarefa, foram desenvolvidas as seguintes competências:
\begin{itemize}
    \item Compreensão da arquitetura de Agentes Autônomos
    \item Implementação de servidores compatíveis com o padrão MCP
    \item Desenvolvimento de ferramentas (Tools) para consumo por LLMs
    \item Análise de trade-offs entre integrações estáticas e dinâmicas
    \item Uso de SDKs para orquestração de agentes (ADK)
    \item Debugging de comunicação via transporte stdio/SSE
\end{itemize}



%
% O arquivo de formatação abntex2-alf.bst coloca todas as entradas no formato correto.
%

\bibliographystyle{abntex2-alf}
\bibliography{TEXTO/TEXTO-Bibliografia}


\includepdf[pages=-]{ultimas-paginas.pdf}


\appendix
\chapter{Códigos da Migração de Arquitetura React Native}
\label{apendice:migracao}

Este apêndice contém os códigos fonte utilizados no processo de migração da arquitetura Bridge para a Nova Arquitetura (JSI) do React Native.

\section{Implementação A - Arquitetura Legada}

A implementação abaixo utiliza a arquitetura legada, empregando o módulo NativeModules.

\begin{lstlisting}[language=JavaScript, caption={App.js (Lado JS)}, label={lst:legacy-js}]
import React from 'react';
import { Button, View, NativeModules } from 'react-native';
const { HelloWorldModule } = NativeModules;

const App = () => {
  const handlePressLegacy = async () => {
      const saudacao = await HelloWorldModule.getHelloWorld();
      console.log(saudacao); // Imprime "Olá mundo (Legado via Bridge)"
  };

  return (
    <View style={{ flex: 1, justifyContent: "center" }}>
      <Button title="Testar Bridge Legada" onPress={handlePressLegacy}/>
    </View>
  );
};

export default App;
\end{lstlisting}

\newpage
No Código Nativo (Android - Java):

\begin{lstlisting}[language=Java, caption={HelloWorldModule.java}, label={lst:legacy-java}]
package com.meuapp; // HelloWorldModule.java
import com.facebook.react.bridge.ReactApplicationContext;
import com.facebook.react.bridge.ReactContextBaseJavaModule;
import com.facebook.react.bridge.ReactMethod;
import com.facebook.react.bridge.Promise;

public class HelloWorldModule extends ReactContextBaseJavaModule {
    public HelloWorldModule(ReactApplicationContext context) {
        super(context);
    }

    @Override
    public String getName() {
        return "HelloWorldModule"; // Nome usado no JS
    }

    @ReactMethod // Metodo exposto para o JS
    public void getHelloWorld(Promise promise) {
        promise.resolve("Ola mundo (Legado via Bridge)"); 
    }
}
\end{lstlisting}


\newpage
No código nativo (iOS - Objective-C):

\begin{lstlisting}[language=C++, caption={HelloWorldModule (Objective-C)}, label={lst:legacy-objc}]
#import <React/RCTBridgeModule.h> // HelloWorldModule.h
// O modulo deve implementar o protocolo <RCTBridgeModule>
@interface HelloWorldModule : NSObject <RCTBridgeModule>
@end

#import "HelloWorldModule.h"
@implementation HelloWorldModule
// Expoe o modulo para o JS com o nome "HelloWorldModule"
RCT_EXPORT_MODULE(HelloWorldModule);
// Expoe o metodo getHelloWorld para o JS
RCT_EXPORT_METHOD(getHelloWorld: (RCTPromiseResolveBlock)resolve
                  rejecter: (RCTPromiseRejectBlock)reject)  {
    NSString *saudacao = @"Ola mundo (Legado via Bridge)";
    if (saudacao) {
        resolve(saudacao);
    } else {
        reject(@"error", @"Nao foi possível gerar a saudação", nil);
    }
}
@end
\end{lstlisting}


\newpage
\section{Implementação B - Nova Arquitetura}

\textbf{Definição da Interface de API (Spec - TypeScript)}:

\begin{lstlisting}[language=TypeScript, caption={NativeHelloWorld.ts (Spec)}, label={lst:new-arch-spec}]
// /js/NativeHelloWorld.ts (A "Spec")
import type { TurboModule } from 'react-native/Libraries/TurboModule/TurboModule';
import { TurboModuleRegistry } from 'react-native';
// 1. Definimos a interface
export interface Spec extends TurboModule {
// 2. Definimos um método SÍNCRONO
  getHelloWorld(): string; 
}
// 3. Registramos o módulo (o nome 'RTNHelloWorld' é o nome oficial do componente)
export default TurboModuleRegistry.getEnforcing<Spec>(
  'RTNHelloWorld',
);
\end{lstlisting}


\newpage
\textbf{Implementação Nativa (Android - Kotlin)}:

\begin{lstlisting}[language=Java, caption={RTNHelloWorldModule.java}, label={lst:new-arch-java}]
package com.meuapp; //RTNHelloWorldModule.java

import com.facebook.react.bridge.ReactApplicationContext;
import com.facebook.react.bridge.ReactMethod;
import androidx.annotation.NonNull;

// 1. O Codegen gera 'NativeHelloWorldSpec'
import com.facebook.fbreact.specs.NativeHelloWorldSpec; 

public class RTNHelloWorldModule extends NativeHelloWorldSpec {
  public static final String NAME = "RTNHelloWorld";
  public RTNHelloWorldModule(ReactApplicationContext context) {
        super(context);
    }

  @Override
  @NonNull
  public String getName() {
    return NAME;
  }

  @Override
  public String getHelloWorld() {
    return "Ola mundo (Nova Arquitetura via JSI)";
 }
}
\end{lstlisting}


\newpage
\textbf{Código em JS}:

\begin{lstlisting}[language=JavaScript, caption={App.js (Lado JS com Nova Arquitetura)}, label={lst:new-arch-js}]
// App.js (Lado JS)
import React from 'react';
import { Button, View } from 'react-native';

// 1. Importa diretamente o módulo (conforme a Spec)
import RTNHelloWorld from './js/NativeHelloWorld'; 

const App = () => {

  const handlePressNewArch = () => {
    const saudacao = RTNHelloWorld.getHelloWorld();
    // Imprime "Ola mundo (Nova Arquitetura via JSI)"
    console.log(saudacao); 
  };

  return (
    <View style={{ flex: 1, justifyContent: 'center' }}>
      <Button title="Testar Nova Arquitetura" onPress={handlePressNewArch} />
    </View>
  );
};

export default App;
\end{lstlisting}


\newpage
\textbf{Implementação Nativa (iOS - Objective-C)}:

\begin{lstlisting}[language=C++, caption={RTNHelloWorld (Objective-C)}, label={lst:new-arch-objc}]
// RTNHelloWorld.h
// 1. Importa a Spec gerada pelo Codegen (baseada no seu .ts)
#import <RTNHelloWorld/RTNHelloWorldSpec.h> 

// 2. O nome da classe deve corresponder ao que o JS espera (RTNHelloWorld)
// 3. A classe implementa o protocolo da Spec gerada
@interface RTNHelloWorld : NSObject <NativeHelloWorldSpec> 
@end

// RTNHelloWorld.mm
#import "RTNHelloWorld.h"

@implementation RTNHelloWorld

// 1. Exporta o módulo (o nome é o da classe)
RCT_EXPORT_MODULE(RTNHelloWorld)

// 2. Esta é a implementação SÍNCRONA da Spec
- (NSString *)getHelloWorld
{
    // O retorno é direto, sem Promises, sem Bridge
    return @"Olá mundo (Nova Arquitetura via JSI)";
}

/*
   NOTA: O Codegen e o template de TurboModule cuidam de
   todo o "encanamento" C++ (JSI) para que este método
   possa ser chamado diretamente pelo JS.
*/

@end
\end{lstlisting}

\chapter{Códigos de Integração e APIs}
\label{apendice:api}

Este apêndice contém os códigos fonte relacionados à implementação do API Gateway, Webhooks para agentes conversacionais e servidores MCP.

\section{API Gateway}

\textbf{Especificação OpenAPI (Swagger)}:

\begin{lstlisting}[language=yaml, caption={Trecho da Especificação OpenAPI (Gateway Público)}, label={lst:openapi-spec}]
# Trecho da Especificação OpenAPI (Gateway Público)
paths:
  /v1/produtos/{id}:
    get:
      summary: Retorna detalhes de um produto com suas avaliações
      responses:
        '200':
          content:
            application/json:
              schema:
                type: object
                properties:
                  produtoId: { type: integer }
                  nome: { type: string }
                  preco: { type: number }
                  # Propriedade agregada:
                  mediaAvaliacoes: { type: number, description: "Média calculada pelo Gateway" }
                  totalAvaliacoes: { type: integer }
\end{lstlisting}


\newpage
\textbf{Implementação do Gateway (Node.js/Express)}:

\begin{lstlisting}[language=JavaScript, caption={Exemplo de código no API Gateway}, label={lst:api-gateway}]
const express = require('express');
const axios = require('axios');
const router = express.Router();

const URL_PRODUTOS = 'http://servico-produtos:3001/api';
const URL_AVALIACOES = 'http://servico-avaliacoes:3002/api';

router.get('/v1/produtos/:id', async (req, res) => {
    const produtoId = req.params.id;
    const [produtoResponse, avaliacoesResponse] = await Promise.all([
        axios.get(`${URL_PRODUTOS}/produtos/${produtoId}`),
        axios.get(`${URL_AVALIACOES}/avaliacoes/produto/${produtoId}`)
    ]);

    const produto = produtoResponse.data;
    const avaliacoes = avaliacoesResponse.data;
    const media = avaliacoes.length 
                  ? avaliacoes.reduce((soma, a) => soma + a.nota, 0) / avaliacoes.length 
                  : 0;

    const respostaGateway = { produtoId: produto.id, nome: produto.nome, preco: produto.valor, mediaAvaliacoes: parseFloat(media.toFixed(1)), totalAvaliacoes: avaliacoes.length
    };

    return res.status(200).json(respostaGateway);
});
module.exports = router;
\end{lstlisting}


\newpage
\section{Webhooks e Agentes}

\textbf{Webhook para Rastreamento de Encomendas (Node.js)}:

\begin{lstlisting}[language=JavaScript, caption={Webhook para Rastreamento de Encomendas (Node.js)}, label={lst:webhook}]
const axios = require('axios');
exports.dialogflowWebhook = async (req, res) => {
    const trackingCode = req.body.sessionInfo.parameters.codigo_rastreio;
    
    if (!trackingCode || trackingCode.length < 8) {
      return res.json({
        fulfillmentResponse: { messages: [{ text: {  text: ['Codigo de rastreio invalido'] }}] }});}
          
    const apiUrl = `https://api.transportadora.com/v1/tracking/${trackingCode}`;
    const response = await axios.get(apiUrl, {
      headers: { 'Authorization': `Bearer ${process.env.API_TOKEN}` },
      timeout: 5000 // Timeout de 5 segundos});

    const status = response.data; // Exemplo de retorno: { local: "Curitiba", previsao: "2 dias", status: "em_transito" }

    const mensagem = `Sua encomenda ${trackingCode} esta em ${status.local}. ` +
                     `Previsao de entrega: ${status.previsao}. ` +
                     `Status: ${status.status}.`;

    return res.json({
      fulfillmentResponse: { messages: [{ text: { text: [mensagem] }}]},
      sessionInfo: {parameters: {ultimo_status: status.status, data_consulta: new Date().toISOString()}}});
};
\end{lstlisting}


\newpage
\section{Model Context Protocol (MCP)}

\textbf{Implementação de Servidor MCP (TypeScript)}:

\begin{lstlisting}[language=JavaScript, caption={Implementação de Servidor MCP (TypeScript)}, label={lst:mcp-server}]
import { Server } from "@modelcontextprotocol/sdk/server/index.js";
import { StdioServerTransport } from "@modelcontextprotocol/sdk/server/stdio.js";
import { CallToolRequestSchema, ListToolsRequestSchema } from "@modelcontextprotocol/sdk/types.js";

const server = new Server( { name: "estoque-server", version: "1.0.0" }, { capabilities: { tools: {} } });

server.setRequestHandler(ListToolsRequestSchema, async () => {
  return {
    tools: [{ name: "consultar_estoque", description: "Consulta a quantidade disponivel de um produto pelo ID", inputSchema: { type: "object", properties: { produtoId: { type: "string" } }, required: ["produtoId"]}}]};});

server.setRequestHandler(CallToolRequestSchema, async (request) => {
  if (request.params.name === "consultar_estoque") {
    const { produtoId } = request.params.arguments;
    const estoque = db.find(p => p.id === produtoId)?.qtd || 0;
    
    return { content: [{ type: "text", text: `O produto ${produtoId} possui ${estoque} unidades.`}]};
  }
  throw new Error("Ferramenta nao encontrada");
});

const transport = new StdioServerTransport();
await server.connect(transport);
\end{lstlisting}



\end{document}
